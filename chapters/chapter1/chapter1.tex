\chapter{绪论}

\section{研究背景与意义}

\subsection{高速自主飞行的应用需求}
四旋翼无人机具有垂直起降、悬停与高机动性等优势,已在巡检、搜索救援、应急通信、室内外自主作业等场景中展现出重要价值。随着应用需求从“低速、开阔、静态”逐步走向“高速、密集、动态”,无人机自主飞行系统需要在有限算力与严格控制周期约束下完成实时感知、决策与控制,且在噪声观测、分布偏移和执行延迟条件下保持闭环稳定与安全可靠。

在高速密集障碍环境中,传统模块化“感知--规划--控制”流水线通常由状态估计/建图、路径/轨迹规划、以及低层控制器构成。该范式的工程优势在于可解释性强、模块边界清晰、便于调参与验证;然而在高速情形下,串联计算带来的延迟会被闭环动态放大,同时感知误差、建图误差与规划误差会跨模块累积传播,从而导致轨迹震荡、避障失败甚至碰撞风险显著增加。针对高速飞行的这一核心矛盾,Loquercio 等在 Science Robotics 工作中指出:流水线式分解在低速下有效,但在高速密集环境中会因延迟与误差复合而变得脆弱,并提出端到端映射以降低处理延迟与提升鲁棒性 \cite{Loquercio2021HighSpeedWild}。

\subsection{端到端视觉控制的研究价值}
端到端视觉控制旨在从高维视觉观测直接输出控制指令(或短期轨迹),绕开显式建图与复杂规划,从而减少系统时延并提升在噪声与不确定性条件下的闭环表现 \cite{Loquercio2021HighSpeedWild}. 随着仿真平台、深度学习与强化学习的发展,端到端方法在竞速与敏捷飞行等极限工况中不断刷新能力上限:Kaufmann 等提出的 Swift 系统在真实无人机竞速对抗中达到了人类冠军级别并赢得多场比赛,展示了端到端系统在极限场景下的潜力 \cite{Kaufmann2023SwiftNature}。

然而,端到端方法能否成为可信可用的工程方案,关键不仅取决于网络结构与训练技巧,也取决于其在在线闭环系统中的可部署性与可复现性:包括时序建模能力、流式推理状态一致性、以及安全性/平滑性权衡等问题。特别是当策略包含序列模型(如 LSTM、状态空间模型等)并以流式方式部署时,训练(Batch 序列前向)与部署(Streaming 单步递推)之间的状态管理差异会显著影响行为一致性;若实现中出现不当重置,序列模型将退化为“无记忆策略”,从而引发系统性漂移并污染实验结论。这一问题在机器人端到端控制中往往隐蔽却关键,必须通过工程机制与评测范式予以系统解决。

\subsection{研究意义概述}
综上,高速端到端视觉避障研究具有以下意义:
\begin{itemize}
  \item \textbf{提升复杂环境任务效率与覆盖能力:} 高速安全飞行直接决定无人机在林区、城市狭窄空间、灾后废墟等场景的任务可用性与执行效率。
  \item \textbf{推动端到端方法走向可部署与可复现:} 通过系统化讨论流式推理一致性、状态生命周期管理与部署侧约束,为端到端控制的工程落地提供可信证据链。
  \item \textbf{探索新型高效表征与时序建模范式:} 结构化状态空间模型(如 Mamba)在序列建模效率方面具有吸引力 \cite{Gu2023Mamba};而 MambaVision 将 Mamba 思想进一步引入视觉 backbone \cite{Hatamizadeh2025MambaVisionCVPR},为“空间--时间统一建模”提供新方向。
\end{itemize}

\section{研究问题与关键挑战}

\subsection{高速闭环对延迟与噪声的放大效应}
在高速飞行中,感知噪声、执行延迟与动力学不确定性会通过闭环耦合被显著放大。模块化系统的串联推理延迟会等效为状态预测误差,从而导致控制滞后、避障反应不及时与安全裕度降低。端到端策略虽可减少流水线延迟,但仍需在噪声观测条件下做出稳定可靠决策,并在高速下保持闭环稳定 \cite{Loquercio2021HighSpeedWild}。

\subsection{时序建模与短时历史信息利用}
高速避障并非静态映射问题:策略必须利用短时历史信息来抑制观测噪声、捕捉障碍相对运动趋势并稳定控制输出。传统做法多使用 LSTM/RNN 进行时序聚合,但可能面临长序列训练稳定性、计算瓶颈以及部署状态管理敏感等问题。结构化状态空间模型(SSM)提供了另一条路径:例如 Mamba 提出选择性状态空间模型,强调线性复杂度与高吞吐的序列建模能力 \cite{Gu2023Mamba},为在线控制中的时序建模提供潜在优势。

\subsection{流式推理一致性与状态生命周期管理}
序列模型在部署中通常采用流式递推:每个控制周期输入当前观测并更新内部状态。训练与部署的模式差异会带来一致性风险:训练往往采用定长序列 batch 前向,部署则以单步递推更新;若工程实现中误将状态在每个时间步重置,模型将退化为无记忆策略,从而出现系统性漂移与性能崩坏。该问题不仅影响真实部署安全性,也会污染离线评测结论,必须通过严格的状态生命周期管理与硬防护机制加以解决。

\subsection{安全性与平滑性的权衡}
更敏捷的策略往往能够减少碰撞,但也可能产生更高频的控制指令抖动(command jerk),影响执行器寿命、能耗与飞行平滑性。安全学习领域提出了多种路线,包括在训练中引入安全约束与证书,以及在部署侧对策略输出进行安全滤波或约束修正。Brunke 等对安全学习控制进行了系统综述 \cite{Brunke2022SafeLearningReview};基于控制障碍函数(CBF)的安全强化学习框架也被用于在学习控制中强制满足安全约束 \cite{Cheng2019RLwithCBF};而 MPSC(model predictive safety certification)强调通过 MPC 证书对学习控制输出进行最小修改以满足约束 \cite{Wabersich2018MPSC}。对于高速端到端避障系统,在保证安全性的前提下降低 jerk 并建立可部署的平滑机制,是工程落地的重要环节。

\section{研究内容与技术路线}

\subsection{总体研究目标}
本文面向高速端到端视觉避障任务,目标是在密集障碍环境中实现安全、实时、可复现的闭环控制系统,并重点解决以下问题:
\begin{enumerate}
  \item 如何设计高效的空间表征与时序聚合结构,以提升高速段避障鲁棒性与分布外泛化能力;
  \item 如何保证序列模型在流式部署中的状态一致性,避免因错误状态管理导致无记忆退化与系统性漂移;
  \item 如何在保持安全性的同时控制指令抖动代价,构建部署可用的平滑/约束机制。
\end{enumerate}

\subsection{技术路线概述}
本文采用端到端视觉控制框架:每个控制周期策略接收单目深度观测与轻量状态输入,输出世界坐标系下的速度指令,由仿真器/低层控制器执行形成闭环。为支撑大规模数据生成与可控评测,本文使用高保真仿真平台进行训练与测试。Flightmare 提供了面向无人机研究的高性能仿真基座,可用于规模化生成训练轨迹与执行闭环评测 \cite{Song2021Flightmare}。

在策略网络方面,本文以“空间编码 + 时序聚合 + 控制头”为基本结构。时序模块采用结构化状态空间模型 Mamba \cite{Gu2023Mamba},以提升时序建模能力并支持流式递推推理。进一步地,本文将系统化分析流式部署一致性问题,并实现回合边界级状态管理与硬防护机制,以确保评测可复现与部署可信。最后,针对高敏捷策略可能带来的 jerk 代价,本文在部署侧引入轻量约束模块以改善控制平滑性。

\subsection{后续扩展方向:MambaVision 空间编码器}
除时序建模外,视觉编码器的表征能力与计算效率也直接影响系统性能上限与部署代价。近期 MambaVision 提出混合 Mamba-Transformer 的视觉 backbone,并在 CVPR 2025 发表 \cite{Hatamizadeh2025MambaVisionCVPR}。本文将以此为后续方向:在保持时序模块与部署一致性机制不变的条件下,将空间编码器从 ViT 系列替换为 MambaVision,并系统评估其在高速段安全性、分布外泛化与实时性方面的收益与代价,为“空间--时间统一建模”提供工程证据。

\section{本文主要贡献与创新点}
结合上述研究目标与技术路线,本文拟形成如下主要贡献与创新点(按硕士论文组织方式表述):
\begin{enumerate}
  \item \textbf{高速端到端避障的时序策略设计与评测体系:} 构建面向高速视觉避障的端到端策略网络,并在多速度档与不同障碍分布环境下建立系统评测协议与指标体系。
  \item \textbf{流式部署一致性机制:} 系统分析序列模型在流式推理中的状态一致性问题,提出回合边界级状态生命周期管理与硬防护机制,避免无记忆退化与系统性漂移,提升评测可信度与可复现性。
  \item \textbf{空间编码器升级方向(MambaVision):} 在既有端到端闭环框架下引入 MambaVision 视觉 backbone,探索其在高效表征、实时性与泛化能力方面的可行性与收益边界,为后续工程部署提供依据 \cite{Hatamizadeh2025MambaVisionCVPR}。
\end{enumerate}

\section{论文结构安排}
本文其余章节组织如下:第2章综述相关工作与国内外研究现状;第3章给出问题定义、系统框架与评价指标;第4章介绍策略网络结构与训练方法;第5章讨论流式部署一致性与状态管理机制;第6章给出实验设置与结果分析,并在此基础上进一步讨论 MambaVision 替换空间编码器的可行性与实现方案;最后总结全文并展望未来工作。

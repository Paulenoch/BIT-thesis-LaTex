\chapter{流式部署一致性与状态生命周期管理}

端到端控制系统在部署时需要以流式(streaming)方式运行:每个控制周期仅接收当前观测并输出控制指令。当策略包含序列模型(如LSTM、Mamba等)时,流式推理依赖内部状态的正确持续传播。本章系统分析训练模式与部署模式的语义差异如何导致状态管理错误,揭示"无记忆退化"现象的机理与后果,提出回合边界级状态生命周期管理协议与硬防护机制,并通过定量实验验证其对评测结论可信性的决定性影响。

\section{训练与部署的语义差异}

\subsection{Batch训练模式}

在训练阶段,策略网络以定长序列(本文为$T=150$步)进行前向计算。序列模型接收完整序列$\{\mathbf{x}_1, \mathbf{x}_2, \ldots, \mathbf{x}_T\}$,通过并行扫描(Mamba)或循环展开(LSTM)一次性计算所有时间步的输出。在每条训练轨迹的起始处,内部状态$\mathbf{h}_0$被初始化为零向量,随后在序列内逐步更新。

Batch模式的关键特征是:
\begin{itemize}
  \item 整条序列一次性可见,模型可利用未来上下文(在训练时);
  \item 状态在序列起始初始化、序列内连续传播、序列结束后丢弃;
  \item 通过并行算法实现高效训练。
\end{itemize}

\subsection{Streaming推理模式}

在部署阶段,系统以流式方式运行:每个控制周期仅输入当前时刻的观测$\mathbf{x}_t$,通过递推更新内部状态$\mathbf{h}_t$得到当前输出$\mathbf{y}_t$。这意味着:
\begin{itemize}
  \item 每步仅处理单帧数据($T=1$);
  \item 内部状态必须跨控制周期持续传播;
  \item 模型无法访问未来信息,完全依赖历史状态。
\end{itemize}

\subsection{两种模式的等价性条件}

当且仅当以下条件同时成立时,Batch模式与Streaming模式的输出在数学上严格等价:
\begin{enumerate}
  \item 内部状态$\mathbf{h}_0$的初始化方式一致;
  \item 同一回合内状态的更新不被中断或重置;
  \item 输入序列的内容与顺序一致。
\end{enumerate}
违反上述任一条件(尤其是第二条)即会破坏等价性,导致训练与部署的行为不一致。


\section{错误状态重置导致无记忆退化:机理分析}

\subsection{问题描述}

在工程实现中,一个常见但隐蔽的错误是:在每个控制步或每次推理调用时重置序列模型的内部状态$\mathbf{h}_t$为初始值(通常为零向量)。这种"逐步重置"(Step-wise Reset)模式在某些推理框架的默认配置中可能自动触发,或因开发者对状态管理的疏忽而被引入。

\subsection{退化机理}

当内部状态在每个控制步被重置时,递推方程退化为:
\begin{equation}
  \mathbf{h}_t^{\text{reset}} = \bar{\mathbf{A}} \cdot \mathbf{0} + \bar{\mathbf{B}}_t \mathbf{x}_t = \bar{\mathbf{B}}_t \mathbf{x}_t
  \label{eq:reset_degenerate}
\end{equation}
此时模型输出仅取决于当前帧的输入$\mathbf{x}_t$,完全丧失了对历史信息的记忆能力。序列模型退化为一个\textbf{无记忆策略}(memoryless policy),等价于一个不含时序模块的单帧前馈网络。

\subsection{闭环后果}

无记忆退化在闭环控制中引发以下级联效应:
\begin{enumerate}
  \item \textbf{时序聚合失效}:策略无法利用短时历史信息抑制单帧观测噪声、捕捉障碍相对运动趋势或稳定控制输出;
  \item \textbf{控制指令抖动加剧}:缺乏时序平滑能力导致相邻控制步的输出高度不相关,指令变化幅度增大;
  \item \textbf{系统性漂移}:持续的单帧决策在闭环中累积偏差,无人机逐渐偏离目标航线产生系统性横向漂移;
  \item \textbf{碰撞率急剧上升}:漂移与抖动的叠加最终导致避障失败。
\end{enumerate}

\subsection{问题的隐蔽性}

该问题的危险性在于其隐蔽性:
\begin{itemize}
  \item 在离线评测(如验证集上的MSE)中,逐步重置与正确管理的差异可能不明显,因为离线指标通常基于Batch前向计算;
  \item 在低速或简单场景中,无记忆策略仍可能"勉强工作",掩盖了问题的严重性;
  \item 只有在高速、密集障碍的闭环评测中,退化效应才会充分暴露。
\end{itemize}
这意味着如果不进行严格的状态管理验证,研究者可能在不知情的情况下报告被工程实现细节严重污染的实验结论。


\section{回合边界级状态生命周期协议}

针对上述问题,本文提出并实现回合边界级(Episode-level)状态生命周期管理协议。

\subsection{核心原则}

协议的核心原则为:序列模型的内部状态仅在回合边界进行初始化,回合内保持连续传播。形式化地:
\begin{equation}
  \mathbf{h}_t = \begin{cases}
    \mathbf{0} & \text{若 } t = t_{\text{episode\_start}} \\
    \bar{\mathbf{A}} \mathbf{h}_{t-1} + \bar{\mathbf{B}}_t \mathbf{x}_t & \text{若 } t > t_{\text{episode\_start}}
  \end{cases}
  \label{eq:lifecycle}
\end{equation}

\subsection{实现细节}

算法~\ref{alg:lifecycle}给出了状态生命周期管理的完整实现。

\begin{algorithm}[htbp]
\caption{回合边界级状态生命周期管理}
\label{alg:lifecycle}
\begin{algorithmic}[1]
\Require 策略网络 $\pi$,推理参数 \texttt{inf\_params}
\State \textbf{// 在仿真器 reset 信号触发时调用}
\Procedure{OnEpisodeReset}{}
  \State $\texttt{inf\_params.state} \leftarrow \mathbf{0}$ \Comment{清零内部状态}
  \State $\texttt{inf\_params.seqlen\_offset} \leftarrow 0$ \Comment{重置序列偏移}
\EndProcedure
\State
\State \textbf{// 在每个控制步调用}
\Procedure{OnControlStep}{$\mathbf{x}_t$}
  \State \textbf{assert} 未触发逐步重置标志 \Comment{硬防护}
  \State $\mathbf{y}_t \leftarrow \pi.\text{forward}(\mathbf{x}_t, \texttt{inf\_params})$ \Comment{前向推理}
  \State \Comment{状态由 forward 内部自动更新至 inf\_params}
  \State \Return $\mathbf{y}_t$
\EndProcedure
\end{algorithmic}
\end{algorithm}

关键实现要点包括:
\begin{itemize}
  \item \texttt{inference\_params}对象在回合开始时初始化,此后跨所有控制步持续传递;
  \item \texttt{seqlen\_offset}记录当前回合内的累积步数,用于Mamba内部的位置感知;
  \item 回合内的每次前向推理均读取并更新同一状态对象,确保时序信息的连续传播。
\end{itemize}


\section{硬防护机制与可审计日志}

仅依赖开发者的自觉遵守无法保证状态管理协议在所有场景下被正确执行。本文引入以下硬防护机制:

\subsection{运行时断言}

在每个控制步执行前,运行时断言检查当前是否处于"逐步重置"模式。若检测到非安全模式(如推理框架的默认行为触发了状态重置),且未显式开启调试开关,系统\textbf{直接报错终止}(fail-fast),而非静默地以错误模式继续执行。该设计确保任何状态管理错误都会被立即发现而非在实验结束后才暴露。

\subsection{配置锁定}

评测开始时,将请求配置(包括状态管理模式、回合终止条件、速度档位等)写入日志并锁定。运行过程中任何对关键配置的修改尝试都会触发告警,确保实验过程中配置不被意外覆盖。

\subsection{可审计日志}

每次试验的日志包含以下字段:
\begin{itemize}
  \item 请求配置与实际生效配置的对比记录;
  \item 每个回合的状态重置时刻记录(应仅出现在回合边界);
  \item 推理参数(\texttt{inference\_params})的生命周期事件;
  \item 模型权重文件的哈希值与代码版本号。
\end{itemize}
通过上述日志,事后审计可以验证整个实验过程中状态管理协议是否被正确执行。


\section{实验验证:KeepState与ResetState对比}

为定量验证状态生命周期管理对系统性能的影响,本文设置以下消融实验:
\begin{itemize}
  \item \textbf{KeepState}(正确模式):仅在回合边界重置内部状态,回合内连续传播;
  \item \textbf{ResetState}(错误模式):在每个控制步重置内部状态为零向量。
\end{itemize}

两种模式使用\textbf{完全相同的策略权重}(同一训练好的ViT+Mamba模型),仅状态管理方式不同。实验在相同的环境配置与速度设定下进行。

\subsection{定量结果}

表~\ref{tab:state_ablation_thesis}给出了消融实验的核心结果。

\begin{table}[htbp]
\centering
\caption{流式状态管理消融实验(KeepState vs ResetState)}
\label{tab:state_ablation_thesis}
\zihao{5}
\begin{tabular}{lccc}
\toprule
\textbf{模式} & \textbf{Collision Rate (\%)} & \textbf{Mean Jerk (m/s)} & \textbf{Mean Y Drift (m)} \\
\midrule
Mamba (KeepState)  & 0.0  & 0.198 & 0.022 \\
Mamba (ResetState) & 90.0 & 0.376 & 0.770 \\
\bottomrule
\end{tabular}
\end{table}

结果表明:
\begin{enumerate}
  \item \textbf{碰撞率从0\%跃升至90\%}:逐步重置导致策略完全丧失避障能力,几乎整个飞行过程都处于碰撞状态;
  \item \textbf{指令抖动增加约90\%}:Mean Jerk从0.198上升至0.376,反映了无记忆策略输出的高度不稳定性;
  \item \textbf{系统性横向漂移}:Mean Y Drift从$\SI{0.022}{m}$上升至$\SI{0.770}{m}$,表明策略在缺乏时序信息的情况下产生了持续性的横向偏离。
\end{enumerate}

其中Mean Y Drift定义为回合内横向位置绝对值的时间平均:
\begin{equation}
  \text{Mean Y Drift} = \frac{1}{T} \sum_{t=1}^{T} |y_t|
  \label{eq:y_drift}
\end{equation}
$\SI{0.770}{m}$的平均横向偏移在密集障碍环境中已足以显著增加擦碰与碰撞风险。

\subsection{漂移可视化}

图~\ref{fig:drift_thesis}给出了KeepState与ResetState两种模式下的横向漂移可视化对比。

\begin{figure}[htbp]
\centering
\includegraphics[width=0.85\textwidth]{Image/fig_drift_reset_vs_episode.png}
\caption{流式推理中KeepState与ResetState的漂移对比。ResetState(逐步重置)导致显著的横向漂移趋势,反映出时序模型在无记忆退化下的闭环不稳定行为。}
\label{fig:drift_thesis}
\end{figure}

图~\ref{fig:lateral_drift}进一步展示了横向漂移的累积过程。

\begin{figure}[htbp]
\centering
\includegraphics[width=0.85\textwidth]{Image/fig_f_lateral_drift.png}
\caption{KeepState与ResetState模式下横向漂移的累积对比}
\label{fig:lateral_drift}
\end{figure}


\section{Batch--Streaming等价性验证}

除了通过闭环性能差异间接验证外,本文还提出一种直接的等价性单元测试方法:对同一条轨迹数据,分别以Batch模式和Streaming模式进行前向计算,比较两种模式输出的差异。

具体地,给定一条测试轨迹$\{\mathbf{x}_1, \ldots, \mathbf{x}_T\}$:
\begin{enumerate}
  \item 以Batch模式一次性前向计算,得到输出序列$\{\mathbf{y}_1^{\text{batch}}, \ldots, \mathbf{y}_T^{\text{batch}}\}$;
  \item 以Streaming模式逐步前向计算(初始状态为零向量,回合内连续传播),得到$\{\mathbf{y}_1^{\text{stream}}, \ldots, \mathbf{y}_T^{\text{stream}}\}$;
  \item 计算逐步输出差异:
  \begin{equation}
    \Delta \mathbf{v}_t = \|\mathbf{y}_t^{\text{batch}} - \mathbf{y}_t^{\text{stream}}\|_2
    \label{eq:bs_diff}
  \end{equation}
\end{enumerate}

在正确实现下,$\Delta \mathbf{v}_t$应在浮点精度范围内($< 10^{-5}$)。若$\Delta \mathbf{v}_t$显著偏离零,则表明Streaming模式的状态管理存在问题。该测试可作为持续集成(CI)中的回归测试,在代码变更后自动验证Batch--Streaming等价性。


\section{普适性讨论}

\subsection{不同序列模型的影响}

本文揭示的状态管理问题\textbf{并非Mamba独有},而是所有依赖内部状态进行递推推理的序列模型的通用风险:
\begin{itemize}
  \item \textbf{LSTM/GRU}:隐状态$(\mathbf{h}_t, \mathbf{c}_t)$在流式推理中同样需要跨步传播,逐步重置会导致相同的无记忆退化;
  \item \textbf{Mamba}:选择性状态空间模型的内部状态$\mathbf{h}_t$遵循相同的递推更新规则,状态管理需求与LSTM一致;
  \item \textbf{Transformer}:虽然标准Transformer不依赖递推状态,但如果使用KV-cache进行增量推理,错误的cache管理同样会导致行为异常。
\end{itemize}

\subsection{贡献定位}

本章的贡献定位为:提出一种\textbf{通用的状态生命周期管理范式与防护协议},而非仅针对某一特定模型的工程修复。该范式具有以下普适价值:
\begin{enumerate}
  \item 为端到端控制系统中使用序列模型的研究者提供明确的工程规范;
  \item 通过硬防护机制将"隐蔽的工程Bug"转化为"可检测的运行时错误";
  \item 通过Batch--Streaming等价性测试提供系统化的验证手段;
  \item 通过可审计日志确保实验结论的可追溯性。
\end{enumerate}

\subsection{对评测可信度的启示}

本章的实验结果(碰撞率从0\%到90\%的跃变)深刻说明:在端到端控制研究中,\textbf{工程实现细节可以决定性地影响实验结论}。若研究者在不知情的情况下使用了错误的状态管理模式,可能得出"序列模型无助于避障"甚至"序列模型有害"的错误结论。本文通过严格的状态生命周期管理与硬防护机制,确保本文所有实验结论建立在正确的部署语义之上——性能差异反映的是模型能力差异,而非实现缺陷。

\chapter{全SSM架构探索与全文总结}

\section{本章引言}

前两章分别确立了ViT+Mamba策略网络的有效性(第3章)与流式部署一致性机制(第4章)。在此基础上,一个自然的架构设计问题浮现:能否将空间与时间维度的建模统一到同一架构范式下?

ViT作为空间编码器具有强表征能力,但其$O(n^2)$自注意力复杂度在机载算力受限的部署场景中可能成为瓶颈。近期MambaVision\cite{Hatamizadeh2025MambaVisionCVPR}将Mamba思想引入视觉backbone,在保持高表征能力的同时实现更优的效率--精度权衡。Vision Mamba(Vim)\cite{Zhu2024VisionMamba}则以纯双向SSM替代自注意力进行视觉编码。

若将视觉编码器替换为MambaVision,整体架构将形成"SSM(空间)+ SSM(时间)"的全SSM架构。本章定位为架构统一性的前瞻探索:其价值不仅在于性能比较,更在于考察空间--时间同构建模在端到端视觉避障任务中的可行性与能力边界。


\section{方法:MambaVision集成}

\subsection{MambaVision视觉骨干}

MambaVision\cite{Hatamizadeh2025MambaVisionCVPR}采用分层设计,在不同阶段融合Mamba块与自注意力块。其核心创新包括:
\begin{enumerate}
  \item 面向视觉特征的Mamba模块重设计,使SSM适配二维空间结构;
  \item 系统消融验证各阶段Mamba/Attention配比的最优方案;
  \item 在ImageNet分类、COCO检测等基准上取得优于纯ViT与纯Mamba的效率--精度权衡。
\end{enumerate}

\subsection{接口对齐}

替换视觉编码器需确保输出维度与后续时序模块的接口一致。MambaVision的最终特征经全局池化后对齐至512维,与ViT输出维度一致,随后同样与轻量状态拼接、投影至$d_{\text{model}}=192$维输入Temporal Mamba。


\section{控制变量实验设计}

MambaVision替换实验采用严格的控制变量设计,如表~\ref{tab:control_var}所示。

\begin{table}[htbp]
\centering
\caption{MambaVision替换实验控制变量声明}
\label{tab:control_var}
\zihao{5}
\begin{tabular}{p{3.5cm}p{5.5cm}p{2.5cm}}
\toprule
\textbf{变量类别} & \textbf{具体内容} & \textbf{状态} \\
\midrule
\multicolumn{3}{l}{\textit{保持不变(Fixed)}} \\
\quad 时序模块 & 4层Mamba ($d_{\text{model}}=192$) & Fixed \\
\quad 训练流程 & BC,三阶段课程学习 & Fixed \\
\quad 训练数据 & Spheres环境585条轨迹 & Fixed \\
\quad 评测协议 & 5速度档,10次/档,Spheres+Trees & Fixed \\
\quad 部署一致性 & KeepState协议 & Fixed \\
\quad 控制头 & 线性层$\rightarrow$$\mathbb{R}^3$ & Fixed \\
\midrule
\multicolumn{3}{l}{\textit{唯一变量(Varied)}} \\
\quad 视觉编码器 & ViT $\leftrightarrow$ MambaVision & \textbf{Varied} \\
\bottomrule
\end{tabular}
\end{table}

通过该设计,性能差异可严格归因于视觉编码器的架构差异。评测维度包括四个方面:
\begin{enumerate}
  \item 安全性:碰撞率与碰撞事件次数;
  \item OOD泛化:Trees零样本性能;
  \item 推理效率:推理延迟与GPU显存占用;
  \item 平滑性:Command Jerk变化。
\end{enumerate}


\section{案例研究与实验}

% ============================================================
% TODO: MambaVision实验尚未完成,以下为实验框架与待填充位置
% 实验完成后需补充:
% 1. MambaVision vs ViT 在 Spheres/Trees 各速度档的碰撞率/碰撞次数对比表
% 2. 推理时间与显存占用对比柱状图
% 3. Command Jerk 对比
% 4. 定性分析与讨论
% ============================================================

\subsection{性能对比(安全性与成功率)}

[TODO: 待MambaVision实验完成后补充]

安全性对比结果将汇总于表~\ref{tab:mv_safety}。
\begin{table}[htbp]
\centering
\caption{MambaVision vs ViT 安全性对比(Spheres + Trees)}
\label{tab:mv_safety}
\zihao{5}
\begin{tabular}{lcccccc}
\toprule
 & \multicolumn{5}{c}{\textbf{目标速度 (m/s)}} \\
\cmidrule(lr){2-6}
\textbf{方法} & 3 & 5 & 7 & 9 & 12 \\
\midrule
\multicolumn{6}{l}{\textit{Spheres Collision Rate (\%)}} \\
ViT+Mamba & \textbf{--} & \textbf{--} & \textbf{--} & \textbf{--} & \textbf{--} \\
MambaVision+Mamba & \textbf{--} & \textbf{--} & \textbf{--} & \textbf{--} & \textbf{--} \\
\midrule
\multicolumn{6}{l}{\textit{Trees Collision Rate (\%)}} \\
ViT+Mamba & \textbf{--} & \textbf{--} & \textbf{--} & \textbf{--} & \textbf{--} \\
MambaVision+Mamba & \textbf{--} & \textbf{--} & \textbf{--} & \textbf{--} & \textbf{--} \\
\bottomrule
\end{tabular}
\begin{tablenotes}
\item \zihao{6} \textbf{TODO}:待实验完成后填入精确数值。
\end{tablenotes}
\end{table}

\subsection{部署指标(延迟与显存)}

[TODO: 待MambaVision实验完成后补充]

推理效率对比结果将汇总于表~\ref{tab:mv_efficiency}。

% TODO: 推理延迟/显存柱状图
% TODO: Accuracy-Throughput Pareto图

\begin{table}[htbp]
\centering
\caption{推理效率对比}
\label{tab:mv_efficiency}
\zihao{5}
\begin{tabular}{lccc}
\toprule
\textbf{方法} & \textbf{单步推理 (ms)} & \textbf{GPU显存 (MB)} & \textbf{参数量 (M)} \\
\midrule
ViT+Mamba & \textbf{--} & \textbf{--} & 3.50 \\
MambaVision+Mamba & \textbf{--} & \textbf{--} & \textbf{--} \\
\bottomrule
\end{tabular}
\begin{tablenotes}
\item \zihao{6} \textbf{TODO}:待实验完成后填入精确数值。
\end{tablenotes}
\end{table}

\subsection{分辨率与序列长度扩展(可选)}

[TODO: 可选实验,若时间允许]

该实验旨在验证SSM视觉骨干在更高分辨率或更长序列条件下是否保持线性复杂度优势。


\section{讨论与局限}

\subsection{MambaVision vs ViT的适用场景}

[TODO: 待实验结果出炉后,基于数据讨论以下问题]

\begin{itemize}
  \item 在什么速度/障碍条件下MambaVision更好?
  \item 在什么条件下ViT更稳定?
  \item 效率提升是否足以弥补可能的精度差异?
\end{itemize}

\subsection{未来方向}

\begin{itemize}
  \item 更强的预训练:在大规模视觉数据上预训练MambaVision骨干后迁移到避障任务;
  \item Sim-to-Real迁移:结合域随机化与真实飞行数据校正,部署到真实四旋翼平台;
  \item 安全过滤:叠加CBF\cite{Ames2019CBFSurvey}/MPC安全滤波\cite{Wabersich2018MPSC},提供形式化安全保证;
  \item 动态障碍:扩展评测环境以包含动态障碍物。
\end{itemize}


\section{全文总结与展望}

\subsection{主要工作与核心发现}

本文面向高速端到端视觉避障控制任务,围绕"时序建模升级与训练方法、流式部署一致性、全SSM架构探索"三个核心维度,构建了完整的端到端闭环控制系统,主要贡献如下:

(1)ViT+Mamba高速端到端避障策略(第3章)。构建以ViT空间编码、Mamba时序聚合与线性控制头为核心的策略网络。在5个速度档与Spheres/Trees双环境下,ViT+Mamba在高速段碰撞率与碰撞事件次数显著优于ViT+LSTM基线,且分布外泛化优势同样显著。DAgger闭环数据增强进一步降低碰撞频次,跨试验标准差显著收敛。RACS动态速率限制以低于0.1ms的开销实现Jerk显著降低。

(2)流式部署一致性与状态生命周期管理(第4章)。系统揭示了序列模型在端到端控制流式部署中的关键陷阱:状态管理错误导致碰撞率从0\%飙升至90\%、Mean Y Drift从0.022m增至0.770m。提出回合边界级状态管理协议与硬防护机制(运行时断言、配置锁定、可审计日志),将部署一致性从经验问题转化为可验证问题。

(3)全SSM架构探索(第5章)。在保持时序模块与训练流程完全不变的条件下,将视觉编码器替换为MambaVision,形成空间--时间统一的SSM架构,为理解SSM在视觉--运动控制任务中的能力边界提供实证基础。

\subsection{不足与局限性}

\begin{enumerate}
  \item 仿真环境局限:所有实验在Flightmare仿真平台中完成,尚未进行Sim-to-Real验证;
  \item 静态障碍假设:评测环境中障碍物均为静态,未涉及动态障碍避障;
  \item 单一传感模态:仅使用单目深度图像,未融合RGB、激光雷达或事件相机等多模态信息;
  \item DAgger规模有限:仅3轮迭代,每轮18条轨迹,DAgger的上界有待更大规模验证;
  \item RACS的启发式特性:$\delta_t$基于最小深度启发式规则,缺乏形式化安全保证。
\end{enumerate}

\subsection{未来工作展望}

\begin{enumerate}
  \item 仿真到真实迁移:结合域随机化、系统辨识与真实飞行数据校正,将策略部署到真实四旋翼平台;
  \item 动态障碍应对:扩展评测环境以包含动态障碍物,探索时变环境中的避障能力;
  \item 多模态融合:探索将RGB图像、事件相机等多模态信息融合到视觉编码器中;
  \item RACS安全升级:将启发式规则升级为基于TTC或碰撞概率的数据驱动风险估计,或与CBF/MPSC融合;
  \item DAgger扩展:增加迭代轮次与采集规模,在更困难环境配置下探索收益上界。
\end{enumerate}

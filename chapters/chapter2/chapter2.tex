\chapter{相关工作与国内外研究现状}

\section{模块化自主飞行:感知--规划--控制范式}

\subsection{视觉/视觉惯性里程计与 SLAM}
模块化自主飞行系统通常以可靠的状态估计为基础。视觉 SLAM 与视觉惯性里程计(VIO)在无人机导航中被广泛采用:ORB-SLAM2 是经典开源 SLAM 系统,支持单目/双目/RGB-D 多模态输入,在稀疏特征与回环检测框架下实现了高精度定位与建图 \cite{MurArtal2017ORBSLAM2}. VINS-Mono 则是单目视觉惯性紧耦合估计的代表性工作,强调鲁棒初始化、故障恢复与基于优化的后端框架,具备良好实时性与工程可用性 \cite{Qin2018VINSMONO}。

国内在视觉 SLAM/VIO 方向也有较多综述与工程化研究积累,通常围绕前端特征/直接法、后端优化/滤波、动态环境处理与回环鲁棒性等问题展开。总体而言,模块化方法在结构化环境与中低速条件下具有稳定优势;但在高速密集障碍场景中,流水线延迟与跨模块误差传播会显著削弱系统鲁棒性 \cite{Loquercio2021HighSpeedWild}。

\subsection{路径规划与轨迹规划}
在规划层面,采样规划因适用性强、便于处理复杂约束而被广泛采用。RRT* 与 PRM* 给出了渐近最优采样规划的理论基础,成为后续大量规划研究的核心工具 \cite{Karaman2011SamplingOptimal}. 面向未知或部分已知环境中的高速飞行,规划器往往需要在安全性与速度之间折中:FASTER 提出在未知环境中同时维护快速轨迹与安全回退轨迹(safe backup trajectory),以支持更高速度上限并降低碰撞风险 \cite{Faust2018FASTER}。

国内在三维路径规划、避障与工程应用方面积累深厚,常见路线包括基于 RRT 的改进、智能优化算法(如 PSO 等)与滚动重规划等;这些研究通常强调实时性、平滑性与动力学约束满足。总体上,规划类方法具有明确可解释性与安全回退机制,但在高速与极限工况下仍可能受限于感知延迟、建图不完备与计算预算。

\subsection{高保真仿真平台与可复现实验基座}
端到端学习控制高度依赖规模化数据与可控实验条件,因此高保真仿真平台尤为关键。Flightmare 通过将渲染与物理引擎解耦,并支持在速度与精度之间灵活切换,为无人机学习控制提供了高性能仿真基座 \cite{Song2021Flightmare}。此外,敏捷飞行研究也逐渐形成软硬件一体化平台与系统化验证基准,以促进可复现研究与工程落地。

\section{端到端视觉飞行控制:从模仿学习到强化学习}

\subsection{端到端高速飞行与仿真到现实迁移}
端到端方法通过直接将传感观测映射到控制量或短期轨迹,减少显式建图与规划带来的延迟瓶颈。Loquercio 等在 Science Robotics 的工作中提出端到端高速飞行框架,并通过特权信息专家与噪声建模实现从仿真到真实环境的零样本迁移,展示了在森林、雪地、坍塌建筑等复杂环境中的高速飞行能力 \cite{Loquercio2021HighSpeedWild}。该工作系统阐明了高速场景下模块化流水线的局限,并强调端到端映射对降低延迟与提升鲁棒性的价值。

\subsection{强化学习与竞速:能力上限的推进}
随着深度强化学习的发展,端到端系统在竞速场景中不断接近并超越人类水平。Kaufmann 等提出 Swift 系统,通过在仿真中进行深度强化学习并结合真实数据进行校正,在真实无人机竞速对抗中达到冠军级别表现并获得多场胜利 \cite{Kaufmann2023SwiftNature}。该成果表明,在极限工况下端到端系统具有可观能力上限,同时也反映出系统化仿真、数据闭环与工程实现对最终性能的重要性。

\subsection{模仿学习与闭环分布偏移}
行为克隆(Behavioral Cloning, BC)因训练稳定、样本效率高而常用于端到端控制。然而 BC 的典型问题是闭环分布偏移(covariate shift):策略部署后诱导的状态分布与专家演示分布不一致,误差会随时间累积并可能导致失败。DAgger 通过数据集聚合方式缓解该问题,是序列决策模仿学习中的经典方法之一 \cite{Ross2011DAgger}。在高速避障任务中,利用特权信息专家生成高质量示范数据并配合鲁棒训练与严格评测协议,是提升可靠性的常见策略。

\section{视觉表征与网络结构:CNN、ViT 与 MambaVision}

\subsection{视觉 Transformer(ViT)与表征能力提升}
Transformer 通过自注意力建模全局依赖,在视觉领域形成了 ViT 范式:将图像划分为 patch token 并进行 Transformer 编码,结合大规模数据预训练可获得强表征能力 \cite{Dosovitskiy2020ViT}. 在机器人控制中,ViT 逐渐被用于提升对复杂场景的表征鲁棒性,尤其在纹理变化、遮挡与全局结构关系建模方面具备优势。

\subsection{MambaVision:混合 Mamba-Transformer 视觉骨干}
近期,Hatamizadeh 与 Kautz 提出 MambaVision:一种针对视觉应用定制的混合 Mamba-Transformer backbone,通过重新设计 Mamba 形式以增强视觉特征建模能力,并通过系统消融验证将 ViT 与 Mamba 组件融合的可行性 \cite{Hatamizadeh2025MambaVisionCVPR}. MambaVision 在 CVPR 2025 发表,为视觉 backbone 的效率与性能权衡提供了新路径。

对于高速端到端避障系统而言,空间编码器不仅决定感知表征上限,也直接影响推理延迟与算力占用。将空间编码器替换为 MambaVision 的潜在动机包括:(1)在同等算力预算下提升特征质量与分布外鲁棒性;(2)与后续的时序 SSM 模块形成更统一的“SSM 友好”结构;(3)在部署端实现更好的吞吐/延迟特性。上述收益需通过严格控制变量的实验评测验证,而非先验假设。

\section{时序建模:LSTM、Transformer 与结构化状态空间模型}

\subsection{RNN/LSTM 的流式优势与局限}
在端到端控制中,序列模型常用于融合短期历史信息以抑制噪声与增强稳定性。LSTM/RNN 的优势在于天然支持流式递推推理,但也存在训练稳定性、长序列梯度衰减与计算瓶颈等问题。更重要的是,RNN/LSTM 在部署侧对内部状态生命周期管理高度敏感:错误的重置时机可能导致策略行为与训练语义不一致,从而引发闭环性能退化。

\subsection{结构化状态空间模型(S4、Mamba)}
结构化状态空间模型(SSM)是近年来序列建模的重要方向。S4 通过结构化参数化实现对长序列的高效建模,成为 ICLR 2022 的代表性工作之一 \cite{Gu2022S4}. Mamba 提出选择性状态空间模型,通过使 SSM 参数依赖于输入以增强内容选择性,并给出硬件友好的并行算法,从而在保持线性复杂度的同时获得高吞吐与强序列建模能力 \cite{Gu2023Mamba}。Mamba 在多种模态任务上展示了良好性能,并被广泛关注。

在高速闭环控制场景中,SSM 的线性复杂度与递推特性具有潜在优势;但与此同时,SSM 在流式部署时同样依赖内部状态持续传播,其状态生命周期管理必须严格与回合边界对齐,否则会导致“无记忆退化”并产生系统性漂移。围绕这一问题建立可审计、可复现的评测范式,是将 SSM 用于机器人端到端控制的重要工程前提。

\section{安全性、平滑性与部署侧约束机制}

\subsection{安全学习控制研究概述}
学习型控制的部署安全性是机器人系统落地的核心问题之一。Brunke 等在 Annual Review 的综述系统总结了安全学习控制与安全强化学习的主要路线、挑战与开放问题 \cite{Brunke2022SafeLearningReview}。总体而言,安全方法可分为:(1)训练时约束(在目标函数或策略更新中引入安全惩罚/约束);(2)运行时证书与滤波(对策略输出进行可行性检查与最小修改);(3)基于模型的安全回退与混合控制等。

\subsection{基于控制障碍函数与 MPC 证书的方法}
控制障碍函数(CBF)为系统安全约束提供了可证明的形式化工具,并被用于学习控制以在运行时约束系统状态在安全集内演化。Cheng 等提出 RL-CBF 框架,将 CBF 与强化学习结合以提升安全性 \cite{Cheng2019RLwithCBF}。另一方面,Wabersich 与 Zeilinger 提出 MPSC(model predictive safety certification)框架,强调通过 MPC 可行性证书对学习控制输出进行最小修改以满足约束并提升安全性 \cite{Wabersich2018MPSC}。

\subsection{部署侧平滑与速率限制}
在高速端到端避障中,策略的敏捷性提升可能伴随更高的指令抖动(jerk),从而影响执行器寿命、能耗与可控性。部署侧约束(例如速率限制、平滑滤波、风险自适应约束)以较小计算代价改善控制输出质量,是工程落地常用策略。与训练侧约束相比,部署侧机制具有实现简单、可调可控的优势,但需要在安全性与平滑性之间建立明确的权衡评价方法。

\section{国内外研究现状总结与本文切入点}

\subsection{国外研究趋势}
国外在敏捷飞行与端到端控制方面呈现出“系统化验证 + 真实环境挑战 + 高性能仿真平台支撑”的趋势。端到端高速飞行与野外验证工作推动了方法向真实落地发展 \cite{Loquercio2021HighSpeedWild};强化学习在竞速对抗中达到冠军级别能力上限 \cite{Kaufmann2023SwiftNature};同时视觉 backbone 与序列建模结构持续演进,从 CNN/ViT 走向更高效的新型结构(如 Mamba、MambaVision) \cite{Gu2023Mamba,Hatamizadeh2025MambaVisionCVPR}。

\subsection{国内研究现状}
国内在无人机路径规划、避障与工程应用方面积累深厚,规划与优化类方法具有成熟的工程体系与可解释性优势;在视觉 SLAM/VIO 方向也形成了大量研究与综述工作,为自主导航基础能力提供支撑。近年来,随着深度学习与强化学习的普及,国内对端到端视觉避障的探索逐步增强,但在高速闭环系统化验证、流式部署一致性与可复现评测范式方面仍有进一步提升空间。

\subsection{本文切入点与定位}
结合国内外现状与高速避障的关键挑战,本文的切入点在于:围绕“时序建模能力 + 流式部署一致性 + 部署侧约束”构建端到端闭环系统,并在此基础上探索将空间编码器升级为 MambaVision 的可行性与收益边界。本文强调严格控制变量与可审计实现细节,以确保评测结论可信、可复现,并为后续工程部署提供依据。

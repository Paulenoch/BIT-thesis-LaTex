\chapter{相关工作与国内外研究现状}

\section{模块化自主飞行:感知--规划--控制范式}

模块化导航系统通常由状态估计/建图(Perception)、路径/轨迹规划(Planning)与控制器(Control)三大模块串联构成。其优势在于可解释性强、模块边界清晰、各子系统可独立验证与替换;在结构化环境与中低速条件下,配合成熟的视觉/视觉惯性里程计(VIO)、SLAM与轨迹优化器可以取得可靠表现。然而,正如第1章所述,该范式在高速密集障碍环境中的串联延迟与跨模块误差传播会显著放大系统脆弱性。Loquercio等在野外高速飞行工作中对这一问题给出了非常明确的动机与实验论证\cite{Loquercio2021HighSpeedWild}。本节分别从状态估计、路径规划与仿真平台三个层面综述模块化范式的典型方法与局限。

\subsection{视觉/视觉惯性里程计与SLAM}

状态估计是模块化导航系统的基石,其精度与实时性直接决定后续规划与控制模块的可靠性。视觉SLAM与视觉惯性里程计(VIO)在无人机导航中被广泛采用,根据前端特征提取方式的不同,可分为基于特征点的稀疏方法与基于直接法的半稠密/稠密方法两大类。

在稀疏特征方法中,ORB-SLAM2是最具代表性的开源SLAM系统之一\cite{MurArtal2017ORBSLAM2}。该系统支持单目、双目与RGB-D三种模态输入,通过ORB特征的快速提取与匹配实现鲁棒追踪,并集成回环检测与全局位姿图优化以消除累积漂移。ORB-SLAM2在学术研究与工程原型中被广泛使用,其模块化设计(追踪线程、局部建图线程、回环线程)也成为后续SLAM系统的参考架构。然而,ORB-SLAM2依赖纹理特征进行匹配,在弱纹理、重复纹理或快速运动导致的运动模糊条件下,特征提取与匹配质量会显著下降,从而影响定位精度与系统鲁棒性。

在视觉惯性估计方面,VINS-Mono是单目视觉惯性紧耦合估计的代表性工作\cite{Qin2018VINSMONO}。该系统将视觉特征观测与IMU预积分在非线性优化框架中进行联合估计,并设计了鲁棒的在线初始化、故障检测与恢复机制,使其在资源受限的机载平台上也能实现高精度的实时状态估计。VINS-Mono的紧耦合架构能够有效利用IMU的高频率与互补性来弥补视觉在快速运动时的信息缺失,这使得VIO成为中速自主飞行中较为可靠的状态估计方案。

从视觉SLAM面临的核心挑战来看,主要包括:(1)\textbf{尺度与初始化}——单目SLAM存在尺度模糊,需要通过运动先验或IMU融合来恢复度量尺度;(2)\textbf{动态物体}——动态场景中运动对象的特征会污染位姿估计,需要语义或几何手段进行筛除;(3)\textbf{回环检测}——回环检测的鲁棒性与计算效率直接影响长时间运行的漂移修正能力;(4)\textbf{运动模糊与快速旋转}——在高速飞行中尤为突出,图像模糊与大幅帧间运动会导致特征追踪失败。高翔等在"视觉SLAM研究进展"中从稀疏/半稠密/稠密地图构建、前后端优化架构与开放问题等维度对国内外视觉SLAM研究进行了系统梳理\cite{Gao2019SLAMSurvey}。张弓等在"移动机器人视觉惯性SLAM研究进展"中则从滤波与优化两大路线总结了VIO/VI-SLAM的发展脉络与核心挑战\cite{Zhang2018VIOSLAM}。

总体而言,视觉SLAM/VIO在结构化环境与中低速条件下具有稳定优势;但在高速密集障碍场景中,运动模糊会严重削弱视觉前端质量,串联流水线的推理延迟会等效为状态预测偏差,而建图不完备则使规划器缺少足够信息以做出安全决策。这些因素叠加后,模块化系统的闭环鲁棒性会显著降低\cite{Loquercio2021HighSpeedWild}。

\subsection{路径规划与轨迹规划}

规划模块负责在感知模块构建的地图或障碍表示上生成安全可行的飞行路径或动力学可执行的轨迹。根据问题类型,可分为全局路径规划与局部轨迹优化/避障两个层面。

\subsubsection{全局路径规划:采样方法与渐近最优}

在全局路径规划中,采样方法因适用性强、可处理高维与复杂约束空间而被广泛采用。Karaman与Frazzoli提出的RRT*与PRM*给出了渐近最优采样规划的理论基础\cite{Karaman2011SamplingOptimal},证明在采样数趋于无穷时路径代价收敛至全局最优,为后续大量采样规划工作奠定了理论框架。基于RRT*的改进工作在学术与工程中数量庞大,围绕采样策略、树结构优化、启发式引导与并行化等方面持续发展。

国内在三维路径规划方面同样积累了大量研究。王猛等针对无人机三维路径规划与避障问题,结合改进粒子群优化与滚动策略以实现实时避障与平滑性提升\cite{Wang2020UAVPath},反映了国内学界在智能优化方法与工程可用性方面的典型思路。李志刚等在多旋翼无人机避障航迹规划研究中,围绕RRT类算法改进以满足动力学与代价约束展开工作\cite{Li2019MultiRotorRRT},体现了国内在采样规划工程化方面的持续投入。

\subsubsection{局部避障与安全回退机制}

在局部避障与轨迹优化层面,规划器往往需要在"速度"与"安全"之间进行折中。面向未知或部分已知环境中的高速飞行,安全回退是关键机制之一。FASTER规划器强调在未知环境中同时维护快速探索轨迹与安全回退轨迹(safe backup trajectory),使得在前方未知区域被感知到障碍时,系统能够切换到已验证安全的回退轨迹,从而在提高速度上限的同时降低碰撞风险\cite{Faust2018FASTER}。这种"安全回退+轨迹优化"的设计思想在后续多项工作中得到继承与拓展。

总体上,规划类方法具有明确的可解释性与安全回退机制,在低速至中速结构化场景中表现可靠。但在高速极限工况下,规划器仍可能受限于三个因素:(1)感知延迟导致地图更新滞后;(2)建图不完备使得规划空间信息不足;(3)在线重规划的计算预算受限。这些因素共同构成了端到端方法在高速场景中的替代动机。

\subsection{高保真仿真平台与可复现实验基座}

端到端学习控制高度依赖可规模化数据采集与可控实验条件,因此高保真仿真平台是关键基础设施。

Flightmare是面向四旋翼研究的代表性仿真器之一\cite{Song2021Flightmare}。其设计强调渲染引擎与物理引擎的解耦:物理仿真可以在不启动渲染的情况下以极高速率运行(用于大规模数据生成与策略训练),也可以启动高保真渲染以支持视觉观测的生成与可视化评测。这种灵活性使Flightmare成为端到端视觉避障研究中广泛使用的实验平台。

另一方面,Foehn等提出的Agilicious提供了开放软硬件一体化平台\cite{Foehn2022Agilicious},覆盖从模型预测控制到神经网络控制的系统化验证需求。Agilicious不仅开源了仿真与控制代码,还提供了配套硬件方案,使得敏捷飞行研究具备可复现性与工程闭环能力,是近年敏捷无人机研究的重要基础设施之一。

仿真平台的可用性对于端到端方法尤为重要:训练数据的规模与多样性、评测场景的可控性与可重复性、以及仿真到真实(Sim-to-Real)迁移的可行性,都直接影响研究结论的可信度。本文选择Flightmare作为主要实验平台,利用其高效物理仿真与可配置障碍场景支撑大规模训练与系统评测。


\section{端到端视觉飞行控制:从轻量网络到高速野外与冠军竞速}

端到端控制通常以"视觉到控制(vision-to-control)"为核心,直接回归控制量或短期轨迹,避免复杂建图与规划。其关键难点在于如何在噪声观测与分布偏移下保持鲁棒,以及如何在闭环中避免累积误差导致发散。

\subsection{模仿学习、特权信息专家与Sim-to-Real}

端到端视觉控制的早期探索可追溯到DroNet\cite{Loquercio2018DroNet},该工作采用相对轻量的卷积神经网络从单目图像直接输出转向角与碰撞概率,实现了城市环境中的端到端导航,体现了"网络简化+实时可用"的早期思路。Deep Drone Racing\cite{Kaufmann2018DeepDroneRacing}则通过域随机化在仿真中训练并实现零样本迁移到真实四旋翼竞速环境,展示了"仿真大数据+迁移鲁棒性"的重要路线。CAD2RL\cite{Sadeghi2017CAD2RL}进一步验证了纯合成数据训练后直接迁移到真实室内场景的可能性,为大规模仿真训练提供了早期证据。

更进一步,Loquercio等在Learning High-Speed Flight in the Wild中将端到端方法推进到复杂自然与人造环境中的高速飞行\cite{Loquercio2021HighSpeedWild}。该工作的核心设计包括:(1)使用可获取完整环境信息的"特权信息专家"(Privileged Expert)生成高质量示范轨迹;(2)通过系统化的传感噪声建模与域随机化实现零样本仿真到现实迁移;(3)端到端映射从深度图像直接输出速度指令,绕过显式建图与规划。该工作在森林、雪地、坍塌建筑等多种复杂环境中展示了高速飞行能力,系统阐明了高速场景下模块化流水线的局限,是端到端高速避障研究的重要里程碑。

与"避障"任务不同,Deep Drone Acrobatics\cite{Kaufmann2020DeepDroneAcrobatics}面向的是极限机动动作(如翻滚、循环翻转等),同样使用仿真训练与特权信息示范来获得可迁移策略。该工作体现了端到端方法在高动态控制任务中的潜力与方法论共性:特权信息专家+仿真训练+域随机化的组合已成为端到端无人机控制的标准范式之一。

\subsection{强化学习与竞速:从"可飞"到"超越人类冠军"}

随着深度强化学习的发展,端到端系统在竞速场景中不断接近并超越人类水平。Kaufmann等提出的Swift系统\cite{Kaufmann2023SwiftNature}结合仿真深度强化学习与真实数据校正,在真实无人机竞速对抗中达到了与人类冠军同级甚至胜出的水平。Swift系统的成功依赖于多个关键要素:(1)大规模仿真中的分布式强化学习训练提供了充足的探索与策略优化基础;(2)真实飞行数据用于校正仿真与现实之间的系统性偏差(Sim-to-Real Gap);(3)系统化的工程实现确保了从仿真策略到真实飞行器的可靠部署。

Swift的成功表明,在充分的仿真基础设施与工程支撑下,端到端系统不仅可以在简单场景中替代传统流水线,更能在极端动态条件下展现出超越人类操控的性能上限。这一成果也反映出"仿真学习+真实数据校正"已成为端到端控制从实验室走向真实部署的主流范式。

\subsection{模仿学习方法谱系与闭环分布偏移问题}

在端到端控制的训练方法中,行为克隆(Behavioral Cloning, BC)因训练稳定、样本效率高而被广泛采用。BC的基本思路是将专家示范轨迹作为监督信号,通过回归损失直接训练策略网络。然而,BC的典型问题是闭环分布偏移(covariate shift):训练数据来自专家轨迹的状态分布,而策略部署后诱导的实际状态分布会偏离训练分布,小误差会随时间累积并可能导致系统发散。

Ross等提出的DAgger(Dataset Aggregation)\cite{Ross2011DAgger}通过迭代式数据集聚合来缓解这一问题:在每轮迭代中,使用当前策略在线采集数据并由专家标注,然后将新数据合并到训练集中重新训练策略。DAgger的理论分析表明,通过$T$轮迭代后策略的期望损失上界与训练集上的损失呈线性关系,而非像BC那样随时间步二次增长。DAgger是序列决策模仿学习中最经典的方法之一,后续的许多改进(如SafeDAgger、EnsembleDAgger等)都建立在其框架之上。

在高速避障任务中,利用特权信息专家生成高质量示范数据并配合DAgger类闭环数据增强策略是提升可靠性的常见做法。本文采用BC与特权专家的组合进行训练,并在分布外(OOD)环境上进行零样本测试。为避免通过目标域再训练引入评测偏倚,本文严格采用零样本测试协议:策略仅在训练分布内数据上学习,不以任何形式接触测试环境,从而确保泛化能力的评测结论具备可信性。本文的核心控制变量是"时序建模+部署一致性",而非数据泄漏或训练技巧。


\section{时序建模:RNN/LSTM、Transformer与结构化状态空间模型}

高速避障不是静态映射问题:无人机控制回路必须利用短期历史来抑制观测噪声、预测障碍相对运动趋势并稳定闭环输出,因此时序建模是端到端策略的核心能力之一。本节从RNN/LSTM、Transformer到结构化状态空间模型(SSM)的演进线索综述时序建模方法。

\subsection{RNN/LSTM的流式优势与局限}

循环神经网络(RNN)及其变体长短期记忆网络(LSTM)\cite{Hochreiter1997LSTM}是最早被用于端到端控制中时序信息聚合的模型。LSTM通过门控机制(输入门、遗忘门、输出门)选择性地保留与更新记忆状态,缓解了普通RNN在长序列训练中的梯度衰减问题,使其能够在一定范围内维持对历史信息的有效记忆。

在在线控制场景中,LSTM/RNN的天然优势在于支持流式递推推理:每个时间步只需输入当前观测并更新固定大小的隐状态,计算开销恒定且不随序列长度增长。然而,LSTM也面临明确的局限:(1)\textbf{长期依赖建模受限}——尽管门控机制缓解了梯度衰减,但对于真正的长程依赖关系(如数百步以上),LSTM的记忆能力仍然有限;(2)\textbf{训练效率}——RNN的序列依赖性使其无法像Transformer那样进行完全并行化训练,在长序列上训练速度较慢;(3)\textbf{部署状态管理敏感性}——这是一个在实践中往往被忽视但影响深远的问题。

关于状态管理敏感性,需要特别指出:LSTM在在线部署时高度依赖内部隐状态$(\mathbf{h}_t, \mathbf{c}_t)$的正确传播。训练通常采用固定长度序列的batch前向传播,而在线推理则是逐步更新。如果工程实现中在错误时刻(如每次推理调用时)重置隐状态,模型将退化为"无记忆策略"——即每个时间步仅根据当前单帧观测做出决策,完全丧失了时序聚合能力。这种退化往往不会在离线指标上直接暴露,但会在真实闭环中导致系统性漂移与性能崩坏。本文将在第5章中系统分析这一问题并给出工程解决方案。

\subsection{Transformer与视觉Transformer(ViT)}

Transformer\cite{Vaswani2017Transformer}通过自注意力机制(Self-Attention)建模序列中任意位置对之间的依赖关系,彻底摆脱了RNN的序列递推约束,实现了完全并行化训练。其核心优势在于:(1)对长程依赖的建模能力更强——注意力权重直接连接任意两个位置,无需通过中间状态传递;(2)训练高度并行化——所有时间步可同时计算注意力与前馈输出。然而,自注意力机制的计算复杂度为$O(n^2)$($n$为序列长度),在长序列或高分辨率输入下可能成为计算瓶颈。

在视觉领域,Dosovitskiy等提出的Vision Transformer(ViT)\cite{Dosovitskiy2020ViT}将Transformer范式引入图像识别:将图像划分为固定大小的patch token,经线性映射后输入标准Transformer编码器。结合大规模数据预训练,ViT在多项视觉基准上取得了优异表现,展示了Transformer在视觉表征方面的强大能力。

在四旋翼端到端避障方向,Xing等\cite{Xing2024VisionBackbone}系统比较了CNN、U-Net、循环结构与ViT等多种视觉backbone对避障性能的影响,指出在高速与泛化条件下ViT具备明显优势,且在ViT后加入循环模块(recurrence)能进一步改善时序表现。该工作的输出语义采用世界坐标系下的速度指令,与本文的系统设定具有较高一致性,为本文选择ViT作为空间编码器提供了直接的实验依据。

\subsection{结构化状态空间模型:S4与Mamba}

结构化状态空间模型(Structured State Space Model, SSM)是近年来序列建模领域的重要方向,其核心思想是基于连续时间线性状态空间方程对序列数据进行建模:
\begin{equation}
  \mathbf{h}'(t) = \mathbf{A}\mathbf{h}(t) + \mathbf{B}\mathbf{x}(t), \quad \mathbf{y}(t) = \mathbf{C}\mathbf{h}(t) + \mathbf{D}\mathbf{x}(t)
  \label{eq:ssm}
\end{equation}
其中$\mathbf{h}(t)$为隐状态,$\mathbf{x}(t)$为输入,$\mathbf{y}(t)$为输出,$\mathbf{A}, \mathbf{B}, \mathbf{C}, \mathbf{D}$为系统参数矩阵。

S4(Structured State Spaces for Sequence Modeling)\cite{Gu2022S4}通过对矩阵$\mathbf{A}$的结构化参数化(如HiPPO初始化与对角化分解),使得SSM能够在长序列上高效训练,同时保持$O(n)$的计算复杂度。S4在长程依赖基准(如Long Range Arena)上取得了突破性表现,成为ICLR 2022的代表性工作之一。

在S4的基础上,Gu与Dao提出了Mamba\cite{Gu2023Mamba},引入\textbf{选择性机制}(Selective Mechanism):使SSM的参数$\mathbf{B}, \mathbf{C}$与离散化步长$\Delta$依赖于输入内容,从而增强模型对不同输入的自适应选择能力。此外,Mamba设计了硬件友好的并行扫描算法,在保持线性复杂度$O(n)$的同时实现了高吞吐推理。与Transformer相比,Mamba在长序列上的推理速度可达数倍提升,且显存占用更低。

对于高速闭环控制场景,Mamba具有双重价值。在\textbf{算法层面},选择性SSM能够根据输入内容动态调整状态更新幅度,对短期历史中的关键信息(如障碍距离突变、急转弯前的预兆)进行自适应聚合。在\textbf{工程层面},Mamba天生支持流式递推推理——离散化后的状态更新方程$\mathbf{h}_t = \bar{\mathbf{A}}\mathbf{h}_{t-1} + \bar{\mathbf{B}}\mathbf{x}_t$与LSTM具有相似的递推形式,每个控制周期仅需一次矩阵运算即可完成状态更新。然而,这也意味着Mamba在流式部署时同样依赖内部状态的正确持续传播,其状态生命周期管理必须严格与回合边界对齐。围绕这一问题建立可审计、可复现的评测范式,是将SSM用于机器人端到端控制的重要工程前提。


\section{视觉Backbone演进:CNN$\rightarrow$ViT$\rightarrow$MambaVision}

端到端避障策略的前端视觉编码器决定了对障碍结构、纹理变化、遮挡与噪声的表征上限。视觉backbone的演进经历了从CNN到ViT再到混合架构的过程。

\subsection{卷积神经网络(CNN)}

卷积神经网络\cite{Lecun1998CNN}通过局部感受野、权重共享与层级特征提取建立了视觉表征的基础范式。ResNet\cite{He2016ResNet}引入的残差连接使得训练更深的网络成为可能,在图像分类、目标检测与语义分割等任务上取得了广泛成功。CNN的优势在于:(1)局部归纳偏置使其在小数据量时也具备较好泛化;(2)计算高度并行且部署效率高;(3)工程成熟度高,工具链完善。在早期端到端避障工作中,CNN是默认的视觉编码器选择\cite{Loquercio2018DroNet,Sadeghi2017CAD2RL}。

然而,CNN在全局结构关系建模方面受限于感受野大小:即便通过深层堆叠扩大感受野,信息仍需逐层传递,对于需要全局上下文(如远处障碍与当前运动方向的关系)的场景可能存在不足。

\subsection{视觉Transformer(ViT)的表征优势}

ViT\cite{Dosovitskiy2020ViT}通过将图像切分为patch token并用Transformer编码器处理,以自注意力机制建模任意patch对之间的全局依赖,突破了CNN的感受野限制。在大规模预训练数据支持下,ViT在多项视觉基准上达到或超越了CNN的表现。

对于端到端避障任务,ViT的全局注意力机制在以下场景中具有潜在优势:(1)复杂障碍布局需要全局结构理解(如多个障碍的相对位置关系);(2)纹理变化、光照变化与遮挡条件下需要更鲁棒的表征;(3)分布外场景中需要更强的泛化能力。Xing等\cite{Xing2024VisionBackbone}的实验比较也支持了ViT在高速避障任务中的优势。

然而,ViT的$O(n^2)$自注意力复杂度($n$为patch数量)在高分辨率输入下可能成为推理瓶颈,尤其是在机载算力受限的部署场景中。这构成了进一步探索更高效视觉backbone的动机。

\subsection{MambaVision:混合Mamba-Transformer视觉骨干}

近期,Hatamizadeh与Kautz提出MambaVision\cite{Hatamizadeh2025MambaVisionCVPR}:一种针对视觉应用定制的混合Mamba-Transformer backbone。MambaVision的核心设计包括:(1)对Mamba模块进行面向视觉特征建模的重新设计,使其更适合二维空间结构的处理;(2)通过系统化消融实验验证在不同阶段融合ViT自注意力块与Mamba块的最优配比;(3)在ImageNet分类、COCO检测与ADE20K分割等多项基准上展示了优于纯ViT与纯Mamba方案的效率--精度权衡。MambaVision于CVPR 2025发表,其代码与预训练模型已开源,为工程复现与任务迁移提供了直接入口。

对于本文的高速端到端避障系统而言,将空间编码器替换为MambaVision的动机包括三个方面。首先,在\textbf{效率}方面,MambaVision在同等精度下相较纯ViT可降低计算量与推理延迟,这对机载部署的时间预算至关重要。其次,在\textbf{架构统一性}方面,本文已采用Mamba作为时序聚合模块;若空间编码器也采用MambaVision(其中包含Mamba组件),则整体架构将形成更统一的"SSM友好"结构,有助于减少架构异质性与工程复杂度。最后,在\textbf{泛化能力}方面,MambaVision的混合注意力+SSM设计是否能在分布外场景中提供更鲁棒的表征,是本文需要通过控制变量实验验证的关键假设。上述收益均需通过严格实验评测确认,而非先验假设。


\section{安全性、平滑性与部署侧约束机制}

学习型控制在真实部署中的核心顾虑之一是安全性:策略的输出可能因训练数据分布的局限、模型泛化的不足或环境扰动而产生不安全动作。同时,在高速避障中,更敏捷的策略可能伴随更高频率的控制指令抖动(jerk),影响执行器寿命、能耗与飞行平滑性。

\subsection{安全学习控制研究概述}

Brunke等在Annual Review的综述中系统总结了从学习控制到安全强化学习的主要路线、挑战与开放问题\cite{Brunke2022SafeLearningReview}。总体而言,安全方法可分为三大类别:(1)\textbf{训练时约束}——在目标函数或策略更新中引入安全惩罚、约束优化或拉格朗日对偶方法,使策略在训练过程中就倾向于满足安全约束;(2)\textbf{运行时证书与滤波}——对策略输出进行可行性检查与最小修改,确保实际执行的动作满足安全约束;(3)\textbf{基于模型的安全回退与混合控制}——在学习策略不可信时切换到经验证安全的备份控制器。每种路线各有优劣:训练时约束可能限制策略探索空间,运行时滤波需要可靠的安全集估计,而混合控制则需要安全控制器的设计与切换逻辑。

\subsection{控制障碍函数与MPC证书}

控制障碍函数(Control Barrier Function, CBF)为系统安全约束提供了可证明的形式化工具。CBF定义了一个安全集$\mathcal{C} = \{\mathbf{x} : h(\mathbf{x}) \geq 0\}$,并通过约束$\dot{h}(\mathbf{x}, \mathbf{u}) + \alpha(h(\mathbf{x})) \geq 0$保证系统状态始终保持在安全集内。Cheng等提出RL-CBF框架\cite{Cheng2019RLwithCBF},将CBF安全约束嵌入强化学习的策略优化过程中,使得学习到的策略在追求任务目标的同时满足安全性。

另一方面,Wabersich与Zeilinger提出MPSC(Model Predictive Safety Certification)框架\cite{Wabersich2018MPSC},采用不同的思路:不修改策略训练过程,而是在部署时对策略输出进行MPC可行性证书验证——若策略的候选动作满足约束则直接执行,否则求解一个最小修改优化问题以找到既满足安全约束又尽量接近原始策略输出的动作。MPSC的优势在于对策略训练无侵入性,可作为"即插即用"的安全层。

\subsection{部署侧平滑与速率限制}

在高速端到端避障的实际部署中,策略的敏捷性提升可能伴随更高的指令变化率。过大的jerk不仅影响执行器寿命与能耗,还会导致飞行器姿态振荡,降低可控性。与训练侧约束相比,部署侧平滑机制具有实现简单、可调可控、对策略训练无侵入等工程优势。

常见的部署侧策略包括:(1)\textbf{速率限制}——限制相邻两个控制周期之间控制量变化的最大幅度;(2)\textbf{低通滤波}——对策略输出进行时域平滑,抑制高频抖动;(3)\textbf{风险自适应约束}——根据当前环境风险等级动态调整约束强度(高风险区放宽限制以保留敏捷性,低风险区增强平滑以节省能耗)。本文的RACS(Rate-Adaptive Control Smoother)模块属于第一与第三种路线的结合:当前版本以动态速率限制为核心,根据最小深度观测值调整相邻控制周期间的指令变化上界约束——障碍接近时放宽限制以保留敏捷性,远离障碍时增强平滑以降低抖动。该模块定位为部署侧的轻量平滑增强组件,虽不提供严格可证明的安全性保证,但通过可控的指令变化约束在实际系统中有效改善jerk并增强可用性,后续可扩展为基于碰撞概率或TTC的风险自适应版本。


\section{国内外研究现状总结与本文切入点}

在展开国内外研究现状讨论之前,表~\ref{tab:route_compare}从系统范式、训练方法、时序建模与视觉编码四个维度对本章涉及的主要技术路线进行了横向对比,以帮助读者快速把握各路线的核心特征与适用场景。

\begin{table}[htbp]
\centering
\caption{高速端到端视觉避障相关技术路线对比}
\label{tab:route_compare}
\zihao{5}
\begin{tabular}{p{1.5cm}p{2.8cm}p{2.8cm}p{2.5cm}p{2.5cm}}
\toprule
\textbf{对比维度} & \textbf{路线A} & \textbf{路线B} & \textbf{A的优势} & \textbf{B的优势} \\
\midrule
系统范式 &
模块化(感知--规划--控制) &
端到端(视觉$\to$控制) &
可解释、可验证、模块可替换 &
低延迟、架构简洁、可吸收大规模仿真数据 \\
\midrule
训练方法 &
行为克隆(BC) &
强化学习(RL) &
训练稳定、样本高效、实现简单 &
可在线探索、策略上限更高、可优化长期回报 \\
\midrule
时序建模 &
LSTM/RNN &
SSM(Mamba) &
流式推理天然支持、工程成熟 &
线性复杂度、高吞吐、选择性机制增强内容感知 \\
\midrule
视觉编码 &
ViT &
MambaVision &
全局注意力、强表征、大规模预训练生态 &
效率--精度更优、与时序SSM架构统一、部署友好 \\
\bottomrule
\end{tabular}
\end{table}

\subsection{国外研究趋势}

国际上,敏捷无人机研究在近年呈现出"系统化验证+端到端方法+高性能仿真平台支撑"的清晰趋势:

\begin{itemize}
  \item 端到端高速飞行与野外复杂场景验证推动了方法从实验室走向真实落地\cite{Loquercio2021HighSpeedWild}。特权信息专家+仿真训练+域随机化的组合已成为标准范式。
  \item 强化学习在竞速对抗中达到人类冠军级别\cite{Kaufmann2023SwiftNature},强调"仿真学习+真实数据校正"的工程闭环。
  \item 开源平台(如Agilicious\cite{Foehn2022Agilicious})与高性能仿真器(如Flightmare\cite{Song2021Flightmare})持续降低研究门槛并提升可复现性。
  \item 在网络结构方面,视觉编码从CNN逐步走向ViT\cite{Dosovitskiy2020ViT},并进一步出现MambaVision\cite{Hatamizadeh2025MambaVisionCVPR}等混合骨干;时序建模从LSTM走向结构化状态空间模型(S4\cite{Gu2022S4}、Mamba\cite{Gu2023Mamba}),"更高效的表征与建模"正在成为新的关注焦点。
\end{itemize}

\subsection{国内研究现状}

国内在无人机自主导航相关技术方面积累深厚,以下按三条典型路线进行归纳式综述。

\subsubsection{三维路径规划与轨迹优化工程化}

国内在采样规划与轨迹优化的工程化方面投入持续。在采样规划层面,李志刚等针对多旋翼无人机避障航迹规划提出改进RRT算法,引入动力学约束与代价优化以提升工程可用性\cite{Li2019MultiRotorRRT};王猛等结合改进粒子群优化与滚动策略实现三维路径规划中的实时避障与平滑性提升\cite{Wang2020UAVPath}。在轨迹优化层面,浙江大学高飞团队的系列工作具有代表性:Zhou等提出Fast-Planner,通过B样条参数化与梯度优化实现高效四旋翼轨迹生成\cite{Zhou2019FastPlanner};进一步的EGO-Planner摆脱了对欧氏有符号距离场(ESDF)的依赖,直接以障碍信息驱动梯度优化,在未知环境中实现了更高效的局部规划\cite{Zhou2021EGOPlanner}。何承坤等在四旋翼轨迹优化与跟踪控制综述中系统梳理了国内外在最优控制、微分平坦与数值优化三条技术路线上的研究进展\cite{He2021QuadTrajectory}。总体而言,国内在规划工程化方面的研究活跃度高、成果丰富,但多数工作仍以模块化范式为前提,较少涉及端到端框架下的规划能力。

\subsubsection{视觉SLAM/VIO工程部署}

视觉SLAM与VIO是无人机自主导航的基础能力模块。高翔等在"视觉SLAM研究进展"中从稀疏/半稠密/稠密地图构建、前后端优化架构到开放问题进行了系统梳理\cite{Gao2019SLAMSurvey};张弓等从滤波与优化两大技术路线总结了移动机器人视觉惯性SLAM的发展脉络\cite{Zhang2018VIOSLAM}。值得注意的是,VINS-Mono\cite{Qin2018VINSMONO}虽以英文发表,但由港科大秦通等中国学者主导,其紧耦合VIO方案已成为国内外无人机状态估计的标准参考实现之一。国内在VIO/VI-SLAM的系统综述、算法改进与嵌入式部署方面积累扎实,为模块化导航提供了可靠的状态估计基座;但这些方法在高速运动模糊与极限工况下仍面临本章前述的固有局限。

\subsubsection{学习型避障与端到端方法探索}

国内在学习型避障方面的研究近年来增长迅速。雷志勇等基于深度强化学习探索了视觉避障方法\cite{Lei2020DRLAvoidance},代表了国内将DRL应用于无人机避障的早期工作。陈杰等在"深度强化学习在无人机自主导航中的研究进展"综述中系统总结了DRL从离散动作空间到连续控制、从仿真训练到Sim-to-Real迁移的技术演进,并对国内相关课题组的代表性成果进行了归纳\cite{Chen2023DRLDroneReview}。总体来看,国内在学习型避障方面呈现以下特征:(1)以深度强化学习路线为主,模仿学习方面的工作相对较少;(2)多数工作以低速或中速场景为验证条件,对高速($>\SI{7}{m/s}$)密集障碍环境的系统化闭环验证尚不充分;(3)对流式部署一致性、状态生命周期管理等工程可信性问题的关注较少。这些空白为本文的研究切入提供了明确的定位依据。

\subsection{研究趋势与本文切入点}

综合国内外现状,可以归纳出三个清晰趋势:

\begin{enumerate}
  \item \textbf{更高速、更复杂环境、更强闭环鲁棒性}成为评价标准——方法的有效性需要在高速密集障碍、分布外场景等挑战性条件下得到验证。
  \item \textbf{端到端方法从"能跑"走向"可复现、可审计、可部署"}——尤其是流式推理一致性、状态生命周期管理与工程防护等问题受到越来越多关注。
  \item \textbf{视觉backbone与时序模型同时演进}——ViT推动了表征升级,SSM(S4/Mamba)推动了高效序列建模,MambaVision进一步把SSM思想带入视觉backbone,为"空间--时间统一建模"提供了新方向。
\end{enumerate}

结合上述趋势与高速避障的关键挑战,本文的切入点在于:围绕"时序建模能力+流式部署一致性+部署侧约束"构建端到端闭环系统,并在此基础上探索将空间编码器升级为MambaVision的可行性与收益边界。本文强调严格控制变量与可审计实现细节,以确保评测结论可信、可复现,并为后续工程部署与架构选型提供系统性依据。

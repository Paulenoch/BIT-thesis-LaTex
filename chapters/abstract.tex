%%==================================================
%% abstract.tex for BIT Master Thesis
%% modified by yang yating
%% version: 0.1
%% last update: Dec 25th, 2016
%%==================================================

\begin{abstract}
四旋翼无人机在高速密集障碍环境中的自主避障是机器人领域的关键挑战。传统模块化导航系统在高速条件下面临流水线延迟累积与误差跨模块传播的固有瓶颈,端到端学习控制方法通过将高维视觉观测直接映射为控制指令,为突破上述瓶颈提供了新的技术路径。然而,端到端方法在高速闭环部署中仍面临时序建模能力不足、流式推理状态管理脆弱以及安全性与平滑性冲突等系统性问题。

本文面向高速端到端视觉避障任务,提出以ViT空间编码与Mamba选择性状态空间模型时序聚合为核心的策略网络架构,构建了涵盖训练方法(BC+DAgger闭环数据增强)、部署约束(RACS动态速率限制)与评测协议的完整系统。主要工作与贡献包括以下三个方面:

第一,提出ViT+Mamba端到端策略网络并建立多速度档系统评测体系。在Flightmare仿真平台中,以行为克隆为基础训练范式,在5个速度档与同分布/分布外双环境下进行系统评测。结果表明,Mamba的选择性时序聚合能力使ViT+Mamba在高速段的碰撞率与碰撞事件次数显著优于ViT+LSTM基线,且分布外泛化优势同样显著。在此基础上,引入DAgger闭环数据增强(3轮迭代),在强BC基线之上进一步降低高速段碰撞频次与碰撞持续时间,跨试验行为方差显著收敛,工程部署稳定性大幅提升。同时,设计部署侧RACS动态速率限制模块,以低于0.1ms的计算开销实现Command Jerk的显著降低,安全性基本保持。

第二,揭示序列模型在端到端控制流式部署中的一个关键陷阱:状态管理错误导致的无记忆退化。通过系统对比实验发现,当序列模型的内部状态在每个推理步被错误重置时,碰撞率从0\%飙升至90\%,Mean Y Drift从0.022m增至0.770m——这一后果此前在端到端控制文献中缺乏系统性报道。本文提出回合边界级状态生命周期管理协议与硬防护机制(运行时断言、配置锁定、可审计日志),确保部署一致性与评测结论的可信性。

第三,探索从混合架构(ViT+Mamba)走向全SSM架构(MambaVision+Mamba)的可行性。在保持时序模块与训练流程完全不变的条件下,将视觉编码器替换为MambaVision,形成空间--时间统一的SSM系列架构,为理解SSM在视觉--运动控制任务中的能力边界提供实证基础。

\keywords{端到端视觉避障;选择性状态空间模型;Mamba;流式部署一致性;行为克隆;DAgger}
\end{abstract}

\begin{englishabstract}

Autonomous obstacle avoidance for quadrotor UAVs in high-speed, densely cluttered environments is a critical challenge in robotics. Traditional modular navigation systems suffer from inherent bottlenecks of pipeline latency accumulation and cross-module error propagation under high-speed conditions. End-to-end learning-based control, which directly maps high-dimensional visual observations to control commands, offers a promising alternative. However, such methods still face systematic issues in high-speed closed-loop deployment, including insufficient temporal modeling capability, fragile streaming inference state management, and conflicts between safety and smoothness.

This thesis addresses the high-speed end-to-end visual obstacle avoidance task by proposing a policy network architecture centered on ViT spatial encoding and Mamba selective state space model temporal aggregation, and constructs a complete system encompassing training methods (BC + DAgger closed-loop data augmentation), deployment constraints (RACS dynamic rate limiting), and evaluation protocols. The main contributions are as follows:

First, the ViT+Mamba end-to-end policy network is proposed with a multi-speed systematic evaluation framework. Using behavioral cloning as the base training paradigm in the Flightmare simulation platform, systematic evaluation is conducted across five speed tiers and both in-distribution (Spheres) and out-of-distribution (Trees) environments. Results demonstrate that Mamba's selective temporal aggregation capability yields significantly lower collision rates and collision counts compared to the ViT+LSTM baseline at high speeds, with the out-of-distribution generalization advantage being equally pronounced. Building upon the strong BC baseline, DAgger closed-loop data augmentation (3 iterations) further reduces collision frequency and duration at high speeds, with cross-trial behavioral variance converging significantly. Additionally, the deployment-side RACS dynamic rate limiter achieves substantial Command Jerk reduction with less than 0.1ms computational overhead while maintaining safety.

Second, a critical pitfall in streaming deployment of sequence models for end-to-end control is revealed: erroneous state management leading to memoryless degradation. Through systematic ablation experiments, it is found that when the internal states of sequence models are incorrectly reset at every inference step, the collision rate surges from 0\% to 90\%, and Mean Y Drift increases from 0.022m to 0.770m---a devastating consequence that has lacked systematic reporting in the end-to-end control literature. An episode-boundary state lifecycle management protocol with hard safeguards (runtime assertions, configuration locking, and auditable logging) is proposed to ensure deployment consistency and evaluation credibility.

Third, the feasibility of transitioning from a hybrid architecture (ViT+Mamba) to a fully SSM-based architecture (MambaVision+Mamba) is explored. By replacing the visual encoder with MambaVision while keeping the temporal module and training pipeline unchanged, a spatially-temporally unified SSM architecture is formed, providing empirical evidence for understanding the capability boundaries of SSMs in visual-motor control tasks.

\englishkeywords{End-to-end visual obstacle avoidance; Selective state space model; Mamba; Streaming deployment consistency; Behavioral cloning; DAgger}

\end{englishabstract}

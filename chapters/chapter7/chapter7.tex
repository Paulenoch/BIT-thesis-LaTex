\chapter{平滑性与安全性权衡:部署侧约束模块}

第6章的主结果表明,
ViT+Mamba通过更强的时序建模能力在高速段显著降低了碰撞率。
然而,
更敏捷的时序响应可能伴随更高频率的控制指令变化(Command Jerk),
影响执行器寿命、能耗与飞行平滑性。
本章针对这一安全性与平滑性之间的内在张力,
提出部署侧动态速率限制控制平滑器(Rate-Adaptive Control Smoother, RACS),
给出其数学定义与算法实现,
讨论其在安全学习方法谱系中的定位,
并通过实验验证其效果。

\section{问题动机:敏捷性与指令抖动的内在张力}

图~\ref{fig:jerk_motivation}展示了不同方法在各速度档位下的Command Jerk对比。

\begin{figure}[htbp]
\centering
\includegraphics[width=0.85\textwidth]{Image/fig_jerk_only.png}
\caption{不同方法的Command Jerk随速度变化趋势(误差条表示跨试验离散程度)。ViT+Mamba的jerk在中高速段整体高于ViT+LSTM基线。}
\label{fig:jerk_motivation}
\end{figure}

\begin{enumerate}
从图中可以观察到: \item ViT+Mamba的Command Jerk在中高速段整体高于ViT+LSTM基线,
表明更强的时序响应能力确实伴随着更激进的指令变化;
 \item 这一现象是符合直觉的:更灵敏的策略能够更快速地响应障碍出现,
但快速响应本身意味着控制指令在短时间内的剧烈变化。
 \end{enumerate}

\begin{itemize}
敏捷性与平滑性之间的张力是一个内在矛盾: \item 更激进的避障动作$\rightarrow$更高频率、更大幅度的指令变化$\rightarrow$更高的jerk;
 \item 更平滑的指令输出$\rightarrow$更低的jerk,
但可能延迟避障反应$\rightarrow$更高的碰撞风险。
 \end{itemize}

因此,
需要一种机制在不过度牺牲安全性的前提下降低jerk代价。


\section{RACS算法定义与数学形式}

\subsection{核心约束}

RACS(Rate-Adaptive Control Smoother)的核心机制是对相邻两个控制周期之间的速度指令变化幅度施加动态上界约束:
\begin{equation}
  \|\mathbf{v}_{\text{cmd}} - \mathbf{v}_{\text{prev}}\|_2 \leq \delta_t
  \label{eq:racs_constraint}
\end{equation}
其中$\mathbf{v}_{\text{cmd}}$为最终发布的速度指令,
$\mathbf{v}_{\text{prev}}$为上一控制步发布的速度指令,
$\delta_t$为当前时刻的动态速率上界。

当网络原始输出$\mathbf{v}_{\text{raw}}$满足约束时直接发布;
否则将$\mathbf{v}_{\text{cmd}}$投影至以$\mathbf{v}_{\text{prev}}$为中心、半径为$\delta_t$的$L_2$球面上:
\begin{equation}
  \mathbf{v}_{\text{cmd}} = \begin{cases}
    \mathbf{v}_{\text{raw}} & \text{若 } \|\mathbf{v}_{\text{raw}} - \mathbf{v}_{\text{prev}}\|_2 \leq \delta_t \\
    \mathbf{v}_{\text{prev}} + \delta_t \cdot \dfrac{\mathbf{v}_{\text{raw}} - \mathbf{v}_{\text{prev}}}{\|\mathbf{v}_{\text{raw}} - \mathbf{v}_{\text{prev}}\|_2} & \text{否则}
  \end{cases}
  \label{eq:racs_projection}
\end{equation}

\subsection{动态速率上界$\delta_t$}

\begin{itemize}
与传统静态限幅器不同,
RACS的速率上界$\delta_t$根据当前环境状态动态调整。
当前实现以最小深度观测值$d_{\min,t}$作为环境风险的代理指标: \item 当障碍接近($d_{\min,t}$较小)时,
\textbf{放宽}$\delta_t$以保留敏捷避障能力;
 \item 当远离障碍($d_{\min,t}$较大)时,
\textbf{收紧}$\delta_t$以增强平滑、降低抖动与能耗。
 \end{itemize}

该启发式反映了一个工程直觉:在安全区域无需急剧机动,
平滑飞行即可;
在危险区域则需要保留策略的全部响应能力。

\subsection{算法伪代码}

算法~\ref{alg:racs}给出了RACS的完整实现。

\begin{algorithm}[htbp]
\caption{RACS动态速率限制控制平滑器}
\label{alg:racs}
\begin{algorithmic}[1]
\Require 网络原始输出$\mathbf{v}_{\text{raw}}$,上一步指令$\mathbf{v}_{\text{prev}}$,当前最小深度$d_{\min,t}$
\Ensure 最终发布指令$\mathbf{v}_{\text{cmd}}$
\State 根据$d_{\min,t}$计算动态速率上界$\delta_t$ \Comment{环境越危险,$\delta_t$越大}
\State $\Delta \mathbf{v} \leftarrow \mathbf{v}_{\text{raw}} - \mathbf{v}_{\text{prev}}$
\If{$\|\Delta \mathbf{v}\|_2 \leq \delta_t$}
  \State $\mathbf{v}_{\text{cmd}} \leftarrow \mathbf{v}_{\text{raw}}$ \Comment{未超出限制,直接发布}
\Else
  \State $\mathbf{v}_{\text{cmd}} \leftarrow \mathbf{v}_{\text{prev}} + \delta_t \cdot \frac{\Delta \mathbf{v}}{\|\Delta \mathbf{v}\|_2}$ \Comment{投影至约束球面}
\EndIf
\State \Return $\mathbf{v}_{\text{cmd}}$
\end{algorithmic}
\end{algorithm}

\subsection{计算开销}

RACS的计算仅涉及一次向量差、一次范数计算与一次条件分支,
计算开销\textbf{低于$\SI{0.1}{ms}$},
相比策略网络的推理时间可忽略不计。
该模块完全在部署侧运行,
\textbf{不修改策略网络的训练过程},
因此可作为"即插即用"的后处理组件。


\section{与安全学习方法的关系}

RACS在安全学习方法的谱系中定位为\textbf{部署侧后处理平滑策略}。
如第2章所述,
安全方法可分为三大类:

\begin{enumerate}
  \item \textbf{训练时约束}(如约束优化、拉格朗日对偶):在训练过程中引入安全惩罚,
    使策略在学习阶段就倾向于满足约束。
    优势是不需要运行时修正;
    劣势是可能限制策略探索空间。
     \item \textbf{运行时安全证书与滤波}(如CBF\cite{Cheng2019RLwithCBF}、MPSC\cite{Wabersich2018MPSC}):在策略输出后进行可行性检查与最小修改,
    确保执行动作满足安全约束。
    优势是提供形式化安全保证;
    劣势是需要精确的安全集估计与动力学模型。
     \item \textbf{部署侧后处理}(RACS属于此类):通过简单的指令变化约束改善平滑性。
    优势是实现简单、调参可控、对策略训练无侵入;
    劣势是不提供严格可证明的安全性保证。
     \end{enumerate}

RACS的设计哲学是:在承认不具备形式化安全保证的前提下,
以最小的工程复杂度换取显著的平滑性改善。
这与安全证书类方法(如CBF/MPSC)形成互补——后续可在RACS框架上叠加更严格的安全约束,
形成分层安全架构\cite{Brunke2022SafeLearningReview}。


\section{实验验证}

\subsection{实验设置}

\begin{enumerate}
RACS验证实验对比以下两种配置: \item \textbf{Mamba No RACS}:ViT+Mamba策略,
不施加任何部署侧约束;
 \item \textbf{Mamba + RACS(Ours)}:ViT+Mamba策略 + RACS动态速率限制。
 \end{enumerate} 两种配置使用相同的策略权重,
仅是否启用RACS模块不同。

\subsection{实验结果}

图~\ref{fig:racs_results}给出了RACS在Spheres环境下的验证结果。

\begin{figure}[htbp]
\centering
\includegraphics[width=0.92\textwidth]{Image/fig_dynamicrl_spheres_medium.png}
\caption{RACS在Spheres环境下的验证结果。对比Mamba No RACS、Mamba + RACS(Ours)以及ViT+LSTM基线在不同速度下的安全性与平滑性表现。}
\label{fig:racs_results}
\end{figure}

\begin{enumerate}
实验表明: \item \textbf{Jerk显著降低}:RACS在多数速度档位能够显著降低Command Jerk,
相比No RACS配置具有稳定改善;
 \item \textbf{安全性基本保持}:启用RACS后的碰撞率整体仍保持在较低水平,
表明动态速率限制未对安全性造成严重损害;
 \item \textbf{计算开销可忽略}:RACS的运行时开销低于$\SI{0.1}{ms}$,
不影响控制回路的实时性。
 \end{enumerate}


\section{约束强度与安全性的权衡分析}

\begin{itemize}
RACS中$\delta_t$的取值范围决定了平滑性与安全性之间的权衡点: \item \textbf{$\delta_t$过小}:指令变化被严格限制,
平滑性极好但避障反应被过度压制,
可能导致碰撞率上升;
 \item \textbf{$\delta_t$过大}:约束形同虚设,
退化为无RACS的状态;
 \item \textbf{适中的$\delta_t$}:在保持足够避障敏捷性的同时有效降低不必要的高频抖动。
 \end{itemize}

当前实现通过基于最小深度$d_{\min,t}$的启发式规则动态调整$\delta_t$,
在实验中取得了良好的jerk-safety权衡。
更精细的$\delta_t$调度策略(如基于速度、加速度或碰撞概率的多因素函数)是可能的改进方向。

需要承认的是:RACS作为部署侧后处理模块,
在极端场景下(如$\delta_t$设置过小且障碍突然出现)可能限制策略的紧急避障能力。
因此,
在实际部署中需要谨慎调参,
确保$\delta_t$的下界不会过度压制安全关键的避障动作。


\section{向风险自适应版本扩展的技术路径}

\begin{itemize}
当前RACS版本以最小深度$d_{\min,t}$作为环境风险的代理指标,
其局限性在于: \item 最小深度仅反映最近障碍物的距离,
不包含运动方向与相对速度信息;
 \item 在某些场景下(如障碍在侧方但无人机并不朝其运动),
最小深度可能给出过于保守的风险估计。
 \end{itemize}

后续可向真正的风险自适应版本扩展,
技术路径包括:

\textbf{(1)基于碰撞时间(TTC)的风险估计。
} 通过结合障碍距离与相对速度,
计算碰撞时间$\text{TTC} = d / v_{\text{rel}}$作为更精确的风险度量。
$\delta_t$与TTC成反比:TTC越小(越危急),
$\delta_t$越大(越放宽)。

\textbf{(2)基于碰撞概率的风险估计。
} 利用策略网络的中间特征或独立的风险预测头,
直接估计短期碰撞概率$P(\text{collision} \mid o_t)$。
该概率可作为$\delta_t$的调度信号。

\textbf{(3)学习型速率调度。
} 将$\delta_t$的调度规则本身作为可学习的模块,
通过端到端训练或在线自适应来优化jerk-safety权衡。

上述扩展方向将RACS从"基于启发式规则的工程模块"升级为"数据驱动的自适应安全组件",
与安全学习/安全证书类方法形成更紧密的衔接。
这些扩展构成本文的重要后续工作方向。

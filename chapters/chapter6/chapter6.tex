\chapter{实验设置与结果分析}

本章给出完整的实验设置与评测结果。首先描述实验平台、环境配置与评测协议,然后给出ViT+Mamba与ViT+LSTM的系统对比结果,并从分布外泛化、消融实验、推理延迟与失败案例等维度进行深入分析。随后依次呈现RACS部署侧约束实验、DAgger闭环数据增强实验,以及从混合架构走向全SSM架构的MambaVision探索实验。

\section{实验平台与环境配置}

\subsection{仿真平台}

所有实验在Flightmare高保真仿真平台\cite{Song2021Flightmare}中完成。Flightmare提供高效的物理仿真引擎与可配置的障碍场景,支撑大规模数据生成、策略训练与系统评测。硬件配置为单张NVIDIA RTX 4060 GPU(8GB显存)。

\subsection{障碍环境}

评测环境包含两类障碍分布:
\begin{enumerate}
  \item \textbf{Spheres}(同分布环境):三维空间中随机分布的球体障碍。训练数据在该环境中生成,因此Spheres作为同分布(In-Distribution)测试条件;
  \item \textbf{Trees}(分布外环境):树状结构障碍(细长圆柱体+冠层),几何形态与训练时的球体障碍存在显著差异。所有策略\textbf{均未在Trees环境中训练},因此Trees环境的评测为零样本(Zero-shot)分布外(OOD)测试。
\end{enumerate}


\section{评测协议}

评测协议的核心参数如表~\ref{tab:eval_protocol}所示。

\begin{table}[htbp]
\centering
\caption{评测协议核心参数}
\label{tab:eval_protocol}
\zihao{5}
\begin{tabular}{lc}
\toprule
\textbf{参数} & \textbf{设置} \\
\midrule
目标速度档位 & 3, 5, 7, 9, 12 m/s \\
每档试验次数 & 10次 \\
回合终止距离 & 沿$X$轴 58--60 m \\
超时限制 & 40 s \\
碰撞处理 & 不终止回合,持续记录 \\
状态管理 & KeepState(回合级重置) \\
测试环境 & Spheres(ID) + Trees(OOD) \\
\bottomrule
\end{tabular}
\end{table}

所有评测方法使用相同的环境配置、随机种子与评测协议,确保对比的公平性。评测结果报告各指标在10次试验上的均值$\pm$标准差。


\section{主结果:ViT+Mamba与ViT+LSTM的系统对比}

\subsection{碰撞率与碰撞事件次数}

图~\ref{fig:main_results}汇总了ViT+Mamba与ViT+LSTM在Spheres(同分布)与Trees(分布外)两类环境中,五个速度档位下的碰撞率、碰撞事件次数、指令抖动(Command Jerk)与推理时间。

\begin{figure}[htbp]
\centering
\includegraphics[width=0.92\textwidth]{Image/fig_main_comparison.png}
\caption{ViT+Mamba与ViT+LSTM基线在Spheres/Trees环境下的性能对比。(a)全程碰撞率随速度变化;(b)碰撞事件次数随速度变化;(c)Command Jerk随速度变化;(d)单步推理时间随速度变化。}
\label{fig:main_results}
\end{figure}

主要发现如下:

\textbf{(1)ViT+Mamba在高速段显著优于ViT+LSTM。}
在Spheres环境中,随着速度从$\SI{3}{m/s}$增至$\SI{12}{m/s}$,两种方法的碰撞率均呈上升趋势。但ViT+Mamba在$\SI{9}{m/s}$至$\SI{12}{m/s}$的高速段,碰撞率与碰撞事件次数均明显低于ViT+LSTM基线。这表明Mamba的选择性状态空间建模能力在高速条件下提供了更强的时序聚合优势。

\textbf{(2)分布外泛化优势同样显著。}
在Trees环境(零样本OOD测试)中,ViT+Mamba同样保持了相对于ViT+LSTM的安全性优势,表明该改进并非仅对同分布场景有效,而是具备跨分布的泛化能力。

\begin{table}[htbp]
\centering
\caption{Spheres环境主结果:碰撞率与碰撞事件次数(均值$\pm$std)}
\label{tab:main_spheres}
\zihao{5}
\begin{tabular}{lcccccc}
\toprule
 & \multicolumn{5}{c}{\textbf{目标速度 (m/s)}} \\
\cmidrule(lr){2-6}
\textbf{方法} & 3 & 5 & 7 & 9 & 12 \\
\midrule
\multicolumn{6}{l}{\textit{Collision Rate (\%)}} \\
ViT+LSTM & 0.82 $\pm$ 0.95 & 4.04 $\pm$ 2.42 & 6.93 $\pm$ 2.33 & 7.82 $\pm$ 2.85 & 5.77 $\pm$ 1.77 \\
ViT+Mamba & 0.00 $\pm$ 0.00 & 0.77 $\pm$ 0.90 & 2.72 $\pm$ 2.75 & 2.49 $\pm$ 2.44 & 3.01 $\pm$ 1.64 \\
\midrule
\multicolumn{6}{l}{\textit{Collision Count}} \\
ViT+LSTM & 1.0 $\pm$ 1.00 & 5.0 $\pm$ 2.86 & 7.6 $\pm$ 2.50 & 9.0 $\pm$ 2.57 & 7.6 $\pm$ 1.74 \\
ViT+Mamba & 0.0 $\pm$ 0.00 & 0.6 $\pm$ 0.66 & 2.1 $\pm$ 1.76 & 2.3 $\pm$ 1.49 & 2.5 $\pm$ 1.50 \\
\bottomrule
\end{tabular}
\begin{tablenotes}
\item \zihao{6} 注:每个数值为10次试验的均值$\pm$标准差。
\end{tablenotes}
\end{table}

\begin{table}[htbp]
\centering
\caption{Trees环境主结果:碰撞率与碰撞事件次数(均值$\pm$std)}
\label{tab:main_trees}
\zihao{5}
\begin{tabular}{lcccccc}
\toprule
 & \multicolumn{5}{c}{\textbf{目标速度 (m/s)}} \\
\cmidrule(lr){2-6}
\textbf{方法} & 3 & 5 & 7 & 9 & 12 \\
\midrule
\multicolumn{6}{l}{\textit{Collision Rate (\%)}} \\
ViT+LSTM & 1.22 $\pm$ 1.16 & 3.09 $\pm$ 1.30 & 7.37 $\pm$ 2.92 & 7.94 $\pm$ 2.15 & 4.49 $\pm$ 1.32 \\
ViT+Mamba & 0.00 $\pm$ 0.00 & 0.95 $\pm$ 1.16 & 2.25 $\pm$ 1.26 & 4.24 $\pm$ 1.97 & 3.30 $\pm$ 1.39 \\
\midrule
\multicolumn{6}{l}{\textit{Collision Count}} \\
ViT+LSTM & 1.4 $\pm$ 1.28 & 3.8 $\pm$ 1.66 & 8.4 $\pm$ 2.94 & 9.2 $\pm$ 2.40 & 5.8 $\pm$ 2.09 \\
ViT+Mamba & 0.0 $\pm$ 0.00 & 0.6 $\pm$ 0.66 & 2.3 $\pm$ 1.55 & 2.9 $\pm$ 0.94 & 3.0 $\pm$ 1.26 \\
\bottomrule
\end{tabular}
\begin{tablenotes}
\item \zihao{6} 注:每个数值为10次试验的均值$\pm$标准差。
\end{tablenotes}
\end{table}


\section{分布外泛化分析:Trees环境}

Trees环境的零样本测试对于验证策略的实际应用价值至关重要:真实部署中障碍的几何形态几乎不可能与训练分布完全一致。

从主结果(图~\ref{fig:main_results}与表~\ref{tab:main_trees})可以观察到:
\begin{enumerate}
  \item ViT+Mamba在Trees环境中的碰撞率虽然相比Spheres有所上升(符合预期,因为OOD场景更具挑战性),但仍优于ViT+LSTM基线;
  \item 这表明Mamba的时序聚合能力对障碍形态变化具有一定的内在鲁棒性:策略学到的并非特定障碍形状的记忆,而是更通用的"动态环境中的时序运动模式";
  \item 本文严格采用零样本测试协议——策略从未接触过Trees环境的任何数据,确保泛化评估的公正性。
\end{enumerate}


\section{消融实验}

为验证各设计选择的必要性,本文进行以下消融实验。

\subsection{时序模块消融:Mamba vs LSTM}

ViT+Mamba与ViT+LSTM的主结果对比本身即构成时序模块的消融实验。在保持视觉编码器、训练流程与评测协议完全一致的条件下,性能差异可归因于时序模块的差异。主要结论为:
\begin{itemize}
  \item Mamba在高速段($\geq \SI{9}{m/s}$)的安全性提升最为显著;
  \item 在低速段($\SI{3}{m/s}$),两种方法的差异较小,因为低速下时序建模的需求不如高速迫切。
\end{itemize}

\subsection{Jerk Loss消融}

三阶段课程学习中的Jerk Loss对控制平滑性具有重要贡献。去除Jerk Loss后,策略的Command Jerk显著增加,尤其在高速段。这表明单纯依赖MSE监督损失不足以引导策略输出平滑的控制指令——显式的平滑性约束是必要的。

\subsection{状态管理消融:KeepState vs ResetState}

该消融实验已在第5章详细讨论。核心结论为:逐步重置会使碰撞率从0\%跳升至90\%,Mean Y Drift从$\SI{0.022}{m}$增至$\SI{0.770}{m}$,充分证明了状态生命周期管理的决定性影响。

表~\ref{tab:ablation_summary}汇总了所有消融实验的关键结果。

% TODO: 请根据实验数据补充完整的消融结果数值
\begin{table}[htbp]
\centering
\caption{消融实验汇总}
\label{tab:ablation_summary}
\zihao{5}
\begin{tabular}{lccc}
\toprule
\textbf{消融条件} & \textbf{Collision Rate (\%)} & \textbf{Mean Jerk (m/s)} & \textbf{说明} \\
\midrule
ViT+Mamba (完整) & \textbf{--} & \textbf{--} & 完整系统 \\
ViT+LSTM (替换时序) & \textbf{--} & \textbf{--} & 验证Mamba贡献 \\
去除Jerk Loss & \textbf{--} & \textbf{--} & 验证平滑性约束 \\
ResetState & 90.0 & 0.376 & 验证状态管理 \\
\bottomrule
\end{tabular}
\begin{tablenotes}
\item \zihao{6} 注:\textbf{--} 表示需从实验日志中填入精确数值。ResetState数据来自第5章消融实验。
\end{tablenotes}
\end{table}


\section{推理延迟与系统时序分析}

\subsection{单步推理时间}

ViT+Mamba策略的单步推理时间在RTX 4060 GPU上为毫秒级,满足高速闭环控制的实时性要求。从图~\ref{fig:main_results}(d)可以观察到,推理时间在不同速度档位下基本保持稳定(因为网络结构不随速度变化),为控制回路提供了稳定的时间预算。

\subsection{控制周期分布}

为排除系统负载差异对实验结论的混淆影响,本文记录了所有试验中每个控制步的时间间隔$\Delta t$分布。图~\ref{fig:dt_dist}给出了该分布的统计结果。

\begin{figure}[htbp]
\centering
\includegraphics[width=0.85\textwidth]{Image/fig_dt_distribution.png}
\caption{控制循环周期$\Delta t$的分布统计。该分布用于检验不同方法的系统时序一致性,并排除负载差异造成的混淆因素。}
\label{fig:dt_dist}
\end{figure}

从$\Delta t$分布可以确认:
\begin{enumerate}
  \item 不同方法(ViT+Mamba vs ViT+LSTM)的控制周期分布高度一致,排除了因推理时间差异导致的不公平比较;
  \item $\Delta t$的变异系数较小,表明系统调度的时序抖动在可控范围内;
  \item 上述一致性为本文的性能对比结论提供了系统级别的可信度保障。
\end{enumerate}


\section{失败案例分析}

\subsection{高速窄通道碰撞}

在$\SI{12}{m/s}$的极限速度下,即使ViT+Mamba也会在某些特定场景中发生碰撞。典型的失败模式包括:
\begin{itemize}
  \item \textbf{窄通道场景}:当多个障碍物形成狭窄通道时,策略需要在极短时间内做出精确的横向调整。高速下的反应时间不足可能导致无人机未能及时转向而碰撞通道壁;
  \item \textbf{连续障碍群}:当障碍物密集排列时,避开第一个障碍后可能直接面对第二个障碍,策略需要在两次避障之间快速恢复稳定姿态。
\end{itemize}

% TODO: 补充轨迹可视化图(若有典型碰撞轨迹的截图)

\subsection{OOD环境下的误判模式}

在Trees环境中,策略偶尔出现以下误判模式:
\begin{itemize}
  \item \textbf{树冠层误判}:树状障碍的冠层在深度图中呈现出与球体不同的视觉特征,策略可能对冠层的距离或范围估计不准确;
  \item \textbf{细长障碍遗漏}:树干的细长形态在$60 \times 90$分辨率的深度图中可能仅占少数像素,在高速运动下容易被忽略。
\end{itemize}

上述失败模式提示了后续改进方向:更高分辨率的视觉输入、更精细的训练数据增强(如引入细长障碍物变体)以及DAgger闭环数据收集可能有助于缓解这些问题。

\subsection{时序记忆衰减与恢复延迟}

除上述空间层面的失败模式外,从时序建模角度审视失败案例还可以发现一个潜在机制——Mamba的\textbf{选择性状态衰减}在避障后姿态恢复阶段可能引入额外风险。

Mamba作为选择性状态空间模型,其核心优势在于通过输入相关的选择性机制(Selective Mechanism)动态调节信息的保留与遗忘\cite{Gu2023Mamba}。在正常飞行阶段,该机制使模型能够高效聚合近期时序信息、过滤冗余历史;但在连续避障场景中,这种"选择性遗忘"可能产生以下副作用:

\begin{enumerate}
  \item \textbf{避障后姿态恢复信息衰减过快}:当无人机完成一次急转避障后,需要恢复至稳态飞行姿态。恢复过程的最优控制依赖于"避障动作的幅度与方向"这一历史信息。若Mamba的选择性机制在避障结束后迅速衰减了该历史状态,策略可能无法准确估计当前的姿态偏移量,导致恢复过程振荡或延迟;
  \item \textbf{连续避障中的"余波"传递不足}:在障碍密集排列的场景中,前一次避障动作的"余波"(如残余横向速度、姿态偏角)对下一次避障的初始条件有直接影响。如果Mamba在两次避障之间过早衰减了前一次避障的状态信息,策略可能以不准确的"当前状态估计"进入下一次避障决策,增加碰撞风险;
  \item \textbf{与LSTM显式门控记忆的对比}:LSTM通过遗忘门(forget gate)和输入门(input gate)对细胞状态进行\textbf{显式}的门控管理,遗忘与保留的决策通过可学习的门控参数实现。相比之下,Mamba的选择性衰减是\textbf{隐式}的——衰减速率由输入信号驱动的选择性参数$\Delta$控制,在训练数据未充分覆盖的极端场景下,$\Delta$的行为缺乏显式的"记忆保护"机制。这一差异可能解释了为何ViT+Mamba在部分极端连续避障场景中的恢复速度不如预期。
\end{enumerate}

上述分析表明,Mamba在高速避障任务中的时序记忆管理存在"敏捷响应"与"状态保持"之间的内在张力——这与第4章RACS中识别的"敏捷性与平滑性"张力在本质上是同源的。未来工作可考虑在Mamba的隐状态管理中引入显式的"关键状态保护"机制,或通过更大规模的DAgger数据覆盖连续避障场景来缓解该问题。


\section{RACS部署侧约束实验}

第4章4.9节提出了RACS动态速率限制控制平滑器,作为部署侧即插即用的后处理模块。本节给出RACS的实验验证结果。

\subsection{实验设置}

RACS验证实验对比以下两种配置:
\begin{enumerate}
  \item \textbf{Mamba No RACS}:ViT+Mamba策略,不施加任何部署侧约束;
  \item \textbf{Mamba + RACS(Ours)}:ViT+Mamba策略 + RACS动态速率限制。
\end{enumerate}
两种配置使用\textbf{相同的策略权重},唯一差异在于是否启用RACS模块。评测协议与前述实验一致(Spheres环境,5速度档,每档10次试验)。

\subsection{实验结果}

图~\ref{fig:racs_results}给出了RACS在Spheres环境下的验证结果。

\begin{figure}[htbp]
\centering
\includegraphics[width=0.92\textwidth]{Image/fig_dynamicrl_spheres_medium.png}
\caption{RACS在Spheres环境下的验证结果。对比Mamba No RACS、Mamba + RACS(Ours)以及ViT+LSTM基线在不同速度下的安全性与平滑性表现。}
\label{fig:racs_results}
\end{figure}

实验表明:
\begin{enumerate}
  \item \textbf{Jerk显著降低}:RACS在多数速度档位能够显著降低Command Jerk,相比No RACS配置具有稳定改善。这验证了第4章4.9节提出的动态速率限制机制能够有效抑制不必要的高频指令抖动;
  \item \textbf{安全性基本保持}:启用RACS后的碰撞率整体仍保持在较低水平,表明基于最小深度$d_{\min,t}$的动态$\delta_t$调度在"障碍临近时放宽约束"方面发挥了预期作用,未对安全性造成严重损害;
  \item \textbf{计算开销可忽略}:RACS的运行时开销低于$\SI{0.1}{ms}$,不影响控制回路的实时性,满足第3章提出的实时性要求。
\end{enumerate}

\subsection{RACS实验小结}

RACS以零训练代价(纯部署侧后处理)实现了Jerk的显著降低,同时安全性损失有限。该结果表明,在端到端策略输出后叠加轻量级约束模块是一种工程可行的平滑性—安全性权衡方案。RACS当前基于启发式$\delta_t$调度,后续可扩展为风险自适应版本(基于TTC或碰撞概率),详见第4章4.9.4节的讨论。


\section{DAgger闭环数据增强实验}

前述实验已验证ViT+Mamba在BC基线上相比ViT+LSTM的安全性优势。本节在同一ViT+Mamba架构上引入DAgger闭环数据增强,考察其在BC基线基础上能否进一步降低碰撞频次与碰撞持续时间、提升跨试验行为稳定性、以及在分布外(OOD)场景中保持或改善系统性能。

\subsection{实验设置}

DAgger实验的核心配置如下(详细方法描述见第4章4.8节):
\begin{itemize}
  \item \textbf{初始策略}:ViT+Mamba的BC训练checkpoint(R0),即前述实验中评测的同一模型;
  \item \textbf{迭代轮次}:3轮DAgger(R1: $\beta=0.7$,R2: $\beta=0.3$,R3: $\beta=0.0$),$\beta$为专家混合比例;
  \item \textbf{数据采集}:每轮在Spheres环境中采集18条轨迹,偏重高速段($\SI{9}{m/s}$与$\SI{12}{m/s}$各6条);
  \item \textbf{评测协议}:与前述实验完全一致——Spheres(ID)+ Trees(OOD),5速度档(3/5/7/9/12 m/s),每档10次试验;
  \item \textbf{Trees零样本}:DAgger数据\textbf{仅在Spheres环境中采集},Trees评测严格保持零样本。
\end{itemize}

需要特别说明的是:BC基线本身已是一个强基线(在多数速度档位下成功率趋于饱和)。因此,DAgger实验的重点不在于提升成功率,而在于考察在成功率饱和条件下,碰撞\textbf{频次}与\textbf{持续时间}能否进一步降低、跨试验行为方差能否收敛。

\subsection{主要结果:碰撞事件次数与碰撞持续时间随轮次下降}

图~\ref{fig:dagger_d1}和图~\ref{fig:dagger_d2}分别给出了碰撞事件次数(Collision Count)与平均碰撞持续时间(Mean Collision Duration)随DAgger轮次的变化趋势,聚焦于高速段($\SI{9}{m/s}$与$\SI{12}{m/s}$)。

\begin{figure}[htbp]
\centering
\includegraphics[width=0.88\textwidth]{Image/fig_d1_collision_count_vs_round.png}
\caption{碰撞事件次数随DAgger轮次变化(Spheres环境,$\SI{9}{m/s}$与$\SI{12}{m/s}$)。R0为BC基线,R1--R3为DAgger各轮次。误差条表示10次试验的标准差。}
\label{fig:dagger_d1}
\end{figure}

\begin{figure}[htbp]
\centering
\includegraphics[width=0.88\textwidth]{Image/fig_d2_collision_duration_vs_round.png}
\caption{平均碰撞持续时间随DAgger轮次变化(Spheres环境,$\SI{9}{m/s}$与$\SI{12}{m/s}$)。碰撞持续时间采用"含零"口径(见第4章4.8.3节说明)。}
\label{fig:dagger_d2}
\end{figure}

主要发现如下:

\textbf{(1)碰撞事件次数随轮次持续下降。}
从R0(BC基线)到R3(DAgger第3轮),碰撞事件次数在$\SI{9}{m/s}$与$\SI{12}{m/s}$高速段均呈现下降趋势。这一改善发生在强BC基线之上,表明DAgger的闭环数据增强能够有效弥补BC在高速段的分布偏移缺陷。

\textbf{(2)碰撞持续时间总体下降,R3存在轻微波动。}
从含零口径的均值来看(图~\ref{fig:dagger_d2}),碰撞持续时间随轮次总体呈下降趋势,在R2达到最低水平,R3维持在较低区间但存在轻微回弹。这一非严格单调的现象与两种口径差异有关:含零口径下,"零碰撞回合比例"随轮次上升压低了整体均值;而条件分布口径(仅看$\text{duration}>0$的回合,详见6.8.4节图~\ref{fig:dagger_d7})下,R3中实际发生碰撞的回合的平均持续帧数相比R2有小幅回升。综合来看,碰撞"频次"持续下降(图~\ref{fig:dagger_d1})与"持续时间"总体改善两个维度共同构成安全性提升的证据,但持续时间维度的改善在R3并非严格单调。

\subsection{行为一致性:方差收敛是DAgger的主要收益}

图~\ref{fig:dagger_d3}给出了三项碰撞指标(碰撞事件次数、碰撞持续时间、碰撞率)的跨试验标准差随DAgger轮次的演化趋势。

\begin{figure}[htbp]
\centering
\includegraphics[width=0.88\textwidth]{Image/fig_d3_stability_evolution.png}
\caption{碰撞指标跨试验标准差随DAgger轮次的演化。左:Collision Count std;中:Collision Duration std;右:Collision Rate std。标准差下降反映了策略行为从"有时好有时差"收敛为"稳定表现"。注意:Duration std采用含零口径(与图~\ref{fig:dagger_d2}一致),因此其趋势可能与条件分布口径(图~\ref{fig:dagger_d7})有所不同。}
\label{fig:dagger_d3}
\end{figure}

跨试验标准差的变化揭示了DAgger在本实验条件下的一个关键收益维度:\textbf{行为一致性提升}。三项指标的std演化分别体现以下特征:
\begin{itemize}
  \item \textbf{Collision Count std}(左子图):碰撞事件次数的跨试验标准差随轮次持续下降,表明策略在"发生多少次碰撞"这一维度上的行为波动性显著收敛;
  \item \textbf{Collision Duration std}(中子图):碰撞持续时间的标准差总体下降但非严格单调,这与6.8.2节的含零口径均值趋势一致——R3中零碰撞回合比例上升压低了含零均值,但条件口径下的持续时间存在轻微波动,影响std的单调性;
  \item \textbf{Collision Rate std}(右子图):碰撞率的标准差随轮次收敛,表明策略在帧级碰撞频率上的表现更加稳定。
\end{itemize}

综合来看:
\begin{itemize}
  \item BC基线策略(R0)虽然在均值上已具备较好的安全性,但在10次独立试验间存在较大的表现波动——某些试验碰撞极少,另一些试验碰撞较多;
  \item DAgger迭代通过引入当前策略诱导的闭环数据,使训练分布逐步覆盖策略实际访问的状态,减少了策略在边界状态下的不确定行为;
  \item 标准差下降意味着策略输出更加可预测、更稳定,这一特性在工程部署中的价值甚至超过均值改善——因为\textbf{稳定的次优策略优于不稳定的"平均最优"策略}。
\end{itemize}

该发现也解释了为何仅18条/轮的有限新增数据即可产生可观收益:DAgger数据针对性地覆盖了BC策略的薄弱状态区域,即使数据量少,也能有效减少这些区域的行为不确定性。

\subsection{分布可视化}

为防止"仅看均值"可能掩盖的分布信息,图~\ref{fig:dagger_d6}和图~\ref{fig:dagger_d7}分别给出了碰撞事件次数与碰撞持续时间的逐轮分布。

\begin{figure}[htbp]
\centering
\includegraphics[width=0.88\textwidth]{Image/fig_d6_collision_count_distribution.png}
\caption{碰撞事件次数的逐轮分布。可以观察到分布随轮次逐步集中于低值区域。}
\label{fig:dagger_d6}
\end{figure}

\begin{figure}[htbp]
\centering
\includegraphics[width=0.88\textwidth]{Image/fig_d7_collision_duration_distribution.png}
\caption{碰撞持续时间的逐轮分布(条件分布,仅统计$\text{duration} > 0$的回合)。注意该图采用条件分布口径,主文结论采用"含零"口径,两者差异见第4章4.8.3节说明。}
\label{fig:dagger_d7}
\end{figure}

需要明确两种口径的差异:
\begin{itemize}
  \item \textbf{主文结论口径(含零)}:所有回合参与统计,未发生碰撞的回合记为$\text{duration}=0$。该口径用于计算跨回合均值与标准差,反映整体安全水平;
  \item \textbf{分布图口径(条件分布,$\text{duration}>0$)}:仅统计实际发生碰撞的回合的持续时间分布。该口径用于分析"一旦碰撞发生,持续多久",提供更细粒度的碰撞行为特征。
\end{itemize}

分布可视化为均值层面的发现提供了更细粒度的解释:碰撞事件次数的分布随轮次逐步集中于低值区域,高碰撞回合的比例下降(图~\ref{fig:dagger_d6});碰撞持续时间的条件分布在R2达到最紧凑状态,R3中虽然中位数维持在较低水平,但箱体较R2略有展宽(图~\ref{fig:dagger_d7}),与6.8.2节含零口径均值的轻微波动一致。这种非严格单调现象在有限数据增量下属于合理范围。

\subsection{BC与DAgger-final的对比:同分布与分布外性能}

图~\ref{fig:dagger_d5}给出了BC基线(R0)与DAgger最终轮(R3)在Spheres(ID)与Trees(OOD)两类环境下各速度档位的碰撞率对比。

\begin{figure}[htbp]
\centering
\includegraphics[width=0.88\textwidth]{Image/fig_d5_collision_rate_vs_speed.png}
\caption{BC基线与DAgger-R3在Spheres(ID)与Trees(OOD)环境下的碰撞率对比(各速度档)。}
\label{fig:dagger_d5}
\end{figure}

图~\ref{fig:dagger_d4}进一步给出了Trees(OOD)环境下的多维指标对比,包括Command Jerk、完成时间与平均前向速度。

\begin{figure}[htbp]
\centering
\includegraphics[width=0.92\textwidth]{Image/fig_d4_trees_ood_multi.png}
\caption{Trees OOD环境下BC与DAgger-R3的多维指标对比(Command Jerk、完成时间、平均前向速度)。}
\label{fig:dagger_d4}
\end{figure}

主要发现如下:
\begin{enumerate}
  \item \textbf{ID环境碰撞率进一步下降}:DAgger-R3在Spheres环境的高速段碰撞率相比BC基线有进一步改善,与前述碰撞频次/持续时间的轮次趋势一致;
  \item \textbf{OOD环境零样本性能保持或改善}:尽管DAgger数据仅在Spheres环境中采集,但在Trees环境的零样本评测中,DAgger-R3并未出现性能退化,部分速度档甚至有所改善。这表明DAgger增强的鲁棒性具有一定的跨分布迁移能力;
  \item \textbf{系统效率与平滑性}:从Trees环境的多维指标来看,DAgger-R3在Command Jerk、完成时间与平均前向速度上均保持或略有改善,说明DAgger数据增强未以牺牲效率或平滑性为代价。
\end{enumerate}

\subsection{DAgger实验小结}

本节的DAgger闭环数据增强实验在ViT+Mamba BC基线之上得到以下增量结论:
\begin{enumerate}
  \item \textbf{碰撞频次与持续时间进一步降低}:3轮DAgger迭代在高速段持续降低碰撞事件次数与碰撞持续时间,在强BC基线上实现了增量安全性提升;
  \item \textbf{跨试验稳定性显著增强}:碰撞指标的标准差随轮次收敛,策略行为从"有波动"过渡为"稳定可预测",工程部署价值显著;
  \item \textbf{OOD性能保持或改善}:仅在Spheres环境中采集DAgger数据的策略,在Trees零样本评测中未退化,且效率与平滑性指标保持稳定。
\end{enumerate}

同时需要客观指出本实验的局限性:
\begin{itemize}
  \item 迭代轮次有限(仅3轮),每轮新增数据量较少(18条轨迹),DAgger的上界有待在更大规模采集下验证;
  \item 在当前环境难度下,BC基线的成功率已趋于饱和,DAgger的主要收益体现在碰撞细粒度指标与方差收敛而非成功率提升;
  \item 更困难的环境配置(更高密度障碍、更极端速度)下DAgger的收益幅度有待探索。
\end{itemize}


\section{从混合架构到全SSM架构:MambaVision探索实验}

前述实验(6.1--6.9节)均基于ViT+Mamba的混合架构:视觉编码由基于注意力机制的ViT完成,时序聚合由基于状态空间模型的Mamba完成。这一"Attention(空间)+ SSM(时间)"的混合范式在实验中取得了良好的安全性与泛化性能,构成了本文的主要贡献基础。

然而,从架构设计的理论视角审视,一个自然的问题是:\textbf{能否将空间与时间维度的建模统一到同一架构范式下?}具体而言,若将视觉编码器从ViT替换为同属SSM系列的MambaVision\cite{Hatamizadeh2025MambaVisionCVPR},形成"SSM(空间)+ SSM(时间)"的\textbf{全SSM架构},系统性能将如何变化?

本节定位为\textbf{架构统一性的前瞻探索}。其价值不仅在于性能比较,更在于考察空间--时间同构建模(spatial-temporal homogeneous modeling)在端到端视觉避障任务中的可行性与潜力。即使实验结果显示全SSM架构的性能提升有限,该探索仍对理解SSM在视觉感知中的能力边界具有理论意义。

\subsection{实验设计}

MambaVision替换实验采用严格的\textbf{控制变量}设计:
\begin{itemize}
  \item \textbf{保持不变}:时序Mamba模块(4层,$d_{\text{model}}=192$)、训练流程(BC、课程学习)、评测协议(5速度档、10次试验、Spheres+Trees)、部署一致性机制(KeepState);
  \item \textbf{唯一变量}:视觉编码器从2-stage ViT替换为MambaVision。
\end{itemize}

通过该控制变量设计,性能差异可严格归因于视觉编码器的架构差异,排除其他混淆因素。

\subsection{评测维度}

替换实验将从以下四个维度进行量化评估:
\begin{enumerate}
  \item \textbf{安全性}:碰撞率与碰撞事件次数在各速度档位下的变化;
  \item \textbf{OOD泛化}:Trees环境中零样本测试的性能对比;
  \item \textbf{推理效率}:单步推理时间与GPU显存占用的变化;
  \item \textbf{平滑性}:Command Jerk是否因视觉编码器变化而改变。
\end{enumerate}

\subsection{预期假设与理论动机}

本实验旨在验证以下三项假设,每项假设背后均有明确的理论动机:
\begin{enumerate}
  \item \textbf{推理效率假设}:MambaVision利用SSM的线性复杂度替代ViT的二次复杂度自注意力,有望在同等或更优安全性下降低推理延迟与显存占用——这对嵌入式部署场景具有直接的工程价值;
  \item \textbf{OOD鲁棒性假设}:MambaVision的混合注意力+SSM设计在OOD场景中可能提供更鲁棒的视觉表征,因为SSM的递归结构对空间局部特征的编码方式有别于全局注意力;
  \item \textbf{架构统一性假设}:MambaVision与时序Mamba形成空间--时间同构的全SSM架构,在工程实现上可共享算子库与优化策略,降低系统复杂度。该假设的核心价值在于\textbf{架构简洁性}而非单纯的性能提升。
\end{enumerate}

% TODO: 待MambaVision实验完成后,在此处补充实验结果(数据表与分析讨论)。
% 实验结果应包含:
% 1. MambaVision vs ViT 在 Spheres/Trees 各速度档的碰撞率/碰撞次数对比表
% 2. 推理时间与显存占用对比
% 3. Jerk 对比
% 4. 定性分析与讨论

上述假设均需通过严格实验评测确认,实验结果将在后续补充。需要强调的是:本节探索的核心贡献在于\textbf{提出并验证全SSM架构在端到端控制中的可行性},而非追求刷新性能数字。无论结果如何,该实验均为理解SSM在视觉-运动控制任务中的能力边界提供了有价值的实证基础。

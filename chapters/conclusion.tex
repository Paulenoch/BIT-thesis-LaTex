%%==================================================
%% conclusion.tex for BIT Master Thesis
%%==================================================


\begin{conclusion}

本文面向高速端到端视觉避障控制任务,
围绕"时序建模升级与训练方法、流式部署一致性、全SSM架构探索"三个核心维度,
构建了完整的端到端闭环控制系统,
并进行了系统化的实验评测与分析。
以下总结本文的主要工作、核心发现与不足之处,
并展望未来研究方向。

\textbf{主要工作与核心发现}

(1)提出ViT+Mamba高速端到端避障策略(第3--4章、第6章)。
本文构建了以ViT空间编码、Mamba选择性状态空间模型时序聚合与线性控制头为核心的端到端策略网络,
采用行为克隆(BC)范式在Flightmare仿真平台中训练。
在Spheres(同分布)与Trees(分布外)两类环境、3--12 m/s五档速度下的系统评测表明,
ViT+Mamba相比ViT+LSTM基线在高速段显著降低碰撞率与碰撞事件次数,
且在零样本OOD场景中仍保持优势。
Mamba的线性复杂度使推理延迟满足高速闭环控制的实时性要求。

(2)揭示流式部署中的无记忆退化现象并提出状态生命周期管理协议(第5章)。
本文系统分析了序列模型在训练(Batch前向)与部署(Streaming逐步递推)之间的语义差异,
揭示了错误的逐步状态重置会使序列模型退化为无记忆策略,
导致碰撞率从0\%跃升至90\%、横向漂移达0.770 m的严重后果。
为此提出回合边界级状态生命周期管理协议,
并实现运行时断言、配置锁定与可审计日志等硬防护机制,
确保评测结论不受实现细节污染。

(3)设计部署侧动态速率限制控制平滑器RACS(第4章方法,
第6章实验)。
针对ViT+Mamba更敏捷时序响应带来的指令抖动(Command Jerk)代价,
本文在部署侧引入RACS模块,
通过动态速率限制约束($\|\mathbf{v}_{\text{cmd}} - \mathbf{v}_{\text{prev}}\|_2 \leq \delta_t$)在不修改训练过程的前提下有效降低jerk,
同时基本保持安全性优势。
RACS的计算开销低于0.1 ms,
可作为即插即用的后处理组件。

(4)在BC基线上实施DAgger闭环数据增强并完成系统评测(第4、6章)。
在ViT+Mamba BC基线的基础上,
本文进一步引入3轮DAgger迭代式闭环数据增强。
实验表明,
DAgger在高速段进一步降低碰撞事件次数与碰撞持续时间,
且跨试验标准差显著收敛——策略行为从"有波动"过渡为"稳定可预测"。
在Trees零样本OOD评测中,
DAgger增强后的策略性能保持或略有改善,
且效率与平滑性指标未退化。

\textbf{不足与局限性}

\begin{enumerate}
本文的研究仍存在以下局限: \item \textbf{仿真环境局限}:所有实验在Flightmare仿真平台中完成,
尚未进行真实环境(Sim-to-Real)验证。
仿真与真实之间的传感噪声、气动效应与延迟特性差异可能影响策略在真实部署中的表现。
 \item \textbf{静态障碍假设}:当前评测环境中的障碍物均为静态,
未涉及动态障碍(如运动物体、其他飞行器)的避障场景。
 \item \textbf{单一传感模态}:策略仅使用单目深度图像作为视觉输入,
未融合RGB图像、激光雷达或事件相机等多模态信息。
 \item \textbf{DAgger规模有限}:本文实施的DAgger仅执行了3轮迭代,
每轮新增18条轨迹,
数据增量较小。
DAgger的安全性提升上界有待在更大采集规模、更多迭代轮次与更高难度环境配置下进一步验证。
 \item \textbf{RACS的启发式特性}:RACS的动态速率上界$\delta_t$基于最小深度的启发式规则,
缺乏形式化安全保证。
 \end{enumerate}

\textbf{未来工作展望}

\begin{enumerate}
基于本文的研究基础,
后续工作将在以下方向展开: \item \textbf{DAgger更大规模扩展}:在当前3轮DAgger的基础上,
增加迭代轮次(5轮以上)、引入更困难的环境配置(更高密度障碍、更极端速度)、扩大每轮采集规模,
系统探索DAgger在高速避障任务中的收益上界。
 \item \textbf{MambaVision视觉编码器替换}:在保持时序Mamba与部署一致性机制不变的条件下,
将空间编码器替换为MambaVision,
量化空间--时间统一SSM架构在安全性、泛化能力与推理效率方面的收益与边界。
 \item \textbf{仿真到真实迁移}:结合域随机化、系统辨识与真实飞行数据校正,
将策略部署到真实四旋翼平台上进行验证。
 \item \textbf{动态障碍应对}:扩展评测环境以包含动态障碍物,
探索策略在时变环境中的避障能力。
 \item \textbf{多模态融合}:探索将RGB图像、事件相机等多模态信息融合到视觉编码器中,
提升感知鲁棒性。
 \item \textbf{RACS向风险自适应升级}:将RACS的启发式规则升级为基于TTC或碰撞概率的数据驱动风险估计,
或与CBF/MPSC等安全证书方法进行融合,
提供更严格的安全保证。
 \end{enumerate}

\end{conclusion}
